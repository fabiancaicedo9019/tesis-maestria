\section{Objetivos del proyecto}
\subsection{Objetivo General}
 Proponer un marco de trabajo que guíe la integración de componentes de Inteligencia Artificial en soluciones de software orientadas al sector de Seguridad y Salud en el Trabajo con base en los requerimientos funcionales.

%Definir buenas prácticas y estrategias para la incorporación de componentes de Inteligencia Artificial en soluciones software orientadas al sector de Seguridad y Salud en el Trabajo.

\subsection{Objetivos específicos}
\begin{enumerate}[label=\textbf{OE\arabic*:}]
    \item Elaborar una lista de buenas prácticas para la incorporación de funcionalidades de Inteligencia Artificial en arquitecturas de productos de software específicos para el sector de Seguridad y Salud en el Trabajo.
    \item Proponer una lista de componentes arquitectónicos potenciales para considerar en la integración de Inteligencia Artificial dentro del desarrollo de soluciones de software.
    \item Definir un conjunto de criterios que permitan especificar el contexto necesario para que la Inteligencia Artificial proporcione respuestas de manera más precisa, enfocándose especialmente en la elaboración y gestión de prompts.
    \item Identificar y proponer posibles adaptaciones arquitectónicas necesarias para la integración de componentes de Inteligencia Artificial en arquitecturas preexistentes de soluciones de software en el ámbito de la seguridad y salud laboral.
\end{enumerate}


\subsection{Resultados esperados}

\begin{enumerate}[label=\textbf{R\arabic*}]
    \item \textbf{(OE1):} Lista de buenas prácticas en la integración de IA en productos de software se enfoca en una gestión de recursos computacionales, la elección de herramientas o librerías optimizadas para la tarea y la selección de modelos de IA específicos para la industria. Este conjunto de prácticas se basa en el análisis de las necesidades computacionales del software para asignar recursos de manera eficiente, la evaluación de herramientas y librerías por su compatibilidad, rendimiento y soporte comunitario, y la elección de modelos de IA que no solo estén en la vanguardia tecnológica, sino que también se alineen estrechamente con los objetivos y procesos industriales específicos del producto.
    \item \textbf{(OE2):} Lista de componentes arquitectónicos para integrar IA durante la fase de diseño de software es una guía para identificar cada elemento necesario y su función dentro del sistema, incluyendo las entradas y salidas de información. Se requiere una comprensión de cómo cada componente interactúa con el sistema existente y cómo el flujo de datos entre estos componentes impulsa las operaciones de IA. Esto ayuda a los diseñadores y desarrolladores a establecer fronteras de responsabilidad y a garantizar que todos los elementos de la arquitectura estén sincronizados para el rendimiento óptimo del sistema.
    \item \textbf{(OE3):} Lista de criterios enfocados en Seguridad y Salud en el Trabajo para configurar adecuadamente el contexto al formular preguntas a un componente de IA, lo que resulta en respuestas más precisas y pertinentes. Estos criterios deben abordar aspectos específicos como la naturaleza del entorno laboral, los riesgos potenciales y las normativas de seguridad aplicables. Al integrar estos elementos en los prompts, la IA puede ajustar sus respuestas para ofrecer información que no solo sea correcta sino también aplicable y segura según las directrices de SST del contexto en cuestión.
    \item \textbf{(OE4):} Lista de posibles adaptaciones arquitectónicas en arquitecturas de software preexistentes para la incorporación de IA aborda la expansión de la capacidad de cómputo, la mejora o adición de interfaces para la integración de nuevos componentes de IA y la actualización de dependencias obsoletas. Estas adaptaciones son fundamentales para superar los desafíos más comunes y para facilitar la transición hacia sistemas mejorados con inteligencia artificial, garantizando que la arquitectura existente se mantenga relevante y pueda soportar las nuevas cargas de trabajo y funcionalidades.
    \item \textbf{(OE4):} Lista de recomendaciones para la modificación o ampliación de arquitecturas de software existentes toma en cuenta las peculiaridades inherentes a los diferentes estilos arquitectónicos en la integración de la IA. Estos estilos incluyen arquitecturas monolíticas, microservicios y basada en servicios, proporcionando una guía para incorporar componentes de IA en el desarrollo evolutivo del software.
\end{enumerate}


% \begin{enumerate}
%     \item Un documento que establece un marco de trabajo, incluyendo prácticas recomendadas para integrar funcionalidades de Inteligencia Artificial en arquitecturas de productos de software, posibles criterios para definir el contexto necesario que permita a la Inteligencia Artificial proporcionar respuestas más precisas, y guías para realizar adaptaciones arquitectónicas en soluciones de software preexistentes.
%     \item Un conjunto de diagramas UML y diagramas del modelo C4, extendiéndose hasta el nivel de componentes, para ilustrar los componentes potenciales a considerar en la integración de Inteligencia Artificial en el desarrollo de soluciones de software. Todos estos diagramas se integrarán y explicarán dentro del marco de trabajo.
%     \item Desarrollo de un prototipo funcional implementado en una plataforma web modular como caso de estudio para validar la propuesta del marco de trabajo.% que sirva para demostrar la aplicación práctica del marco de trabajo desarrollado, facilitando la integración de componentes de Inteligencia Artificial y funcionando 
% \end{enumerate}