\section{Objetivos del proyecto}
\subsection{Objetivo General}
Definir buenas prácticas y estrategias para la incorporación efectiva de componentes de Inteligencia Artificial en soluciones software orientadas al sector de seguridad y salud en el trabajo, atendiendo a los requerimientos funcionales específicos del software.

\subsection{Objetivos específicos}
\begin{enumerate}
    \item Determinar las características y funcionalidades actuales de las soluciones de software en el sector de seguridad y salud en el trabajo que pueden beneficiarse o ser mejoradas con la implementación de Inteligencia Artificial.
    \item Analizar las implicaciones éticas y de privacidad en la incorporación de la Inteligencia Artificial en sistemas de seguridad y salud en el trabajo, identificando posibles riesgos y estrategias de mitigación.
    \item Definir los aspectos relevantes para la gestión eficiente y precisa de prompts en la Inteligencia Artificial, orientados al sector de seguridad y salud en el trabajo.
    \item Identificar las adaptaciones arquitectónicas requeridas para una integración efectiva de componentes de Inteligencia Artificial en las estructuras y funcionalidades ya existentes en soluciones de software para el sector de seguridad y salud en el trabajo.
\end{enumerate}

\subsection{Resultados esperados}
\begin{enumerate}
    \item Un conjunto de principios arquitectónicos base para considerar al construir un aplicativo nuevo en el que haya requisitos relacionados con la Inteligencia Artificial.
    \item Desarrollo de un prototipo funcional que aplique el marco de trabajo documentado propuesto. Este aplicativo será web y modular, diseñado para permitir la integración de un componente de Inteligencia Artificial.
    \item Al finalizar el proyecto, se espera presentar un documento que defina el marco de trabajo y especifique buenas prácticas, convenciones y los requisitos mínimos necesarios para una adecuada integración de aplicaciones que utilicen componentes de Inteligencia Artificial.
\end{enumerate}