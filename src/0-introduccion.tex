\section{Introducción}

En el ámbito global, la Inteligencia Artificial (IA) ha revolucionado diversos sectores, facilitando la toma de decisiones, optimizando procesos y generando soluciones innovadoras para problemas complejos. Particularmente, la IA ha demostrado su potencial en la mejora de sistemas de software diseñados para sectores específicos. El sector de seguridad y salud en el trabajo, especialmente en contextos como el colombiano, se encuentra en la cúspide de esta revolución tecnológica, donde organizaciones como la Corporación Talentum aspiran a liderar iniciativas que generen un impacto positivo en el bienestar laboral.

No obstante, emerge una problemática significativa: a pesar de los avances en IA, su integración efectiva en soluciones de software orientadas a la seguridad y salud en el trabajo presenta desafíos. Estos desafíos abarcan desde aspectos técnicos hasta cuestiones éticas y de privacidad, y demandan una comprensión profunda y enfoques adaptados para asegurar implementaciones exitosas que realmente beneficien a los usuarios finales y a las organizaciones involucradas.

Es imperativo abordar esta problemática, dado que las soluciones adecuadas poseen el potencial de transformar cómo las organizaciones gestionan la seguridad y salud en el trabajo. Una implementación efectiva de IA puede facilitar la identificación temprana de riesgos, optimizar respuestas y promover ambientes de trabajo más seguros y saludables. Por lo tanto, resulta esencial establecer estrategias y métodos claros que permitan maximizar los beneficios de la IA en este sector.

En este contexto, se presenta esta propuesta con el propósito de establecer dichas estrategias y métodos para una efectiva incorporación de componentes de IA en soluciones dirigidas al sector mencionado. Con este fin, no solo se plantea el objetivo de proporcionar directrices claras y evidencia palpable de cómo la IA puede enriquecer este campo, sino que también se espera producir resultados concretos: un conjunto de principios arquitectónicos fundamentales para considerar al diseñar software con requisitos relacionados con la IA; un prototipo funcional web y modular que demuestre la aplicación del marco de trabajo documentado; y finalmente, un documento detallado que defina el marco de trabajo y ofrezca buenas prácticas, convenciones y requisitos esenciales para la adecuada integración de aplicaciones con componentes de IA.

Para lograr una comprensión profunda y enfrentar la problemática destacada, se se realizará una revisión de literatura, fundamentado en una revisión profunda de documentos existentes, el análisis de datos pertinentes y la aplicación de metodologías apropiadas. De esta manera, se proporcionará no solo una base sólida para futuras implementaciones de IA en el ámbito de seguridad y salud laboral, sino también en productos que incorporen IA en sus requerimientos funcionales.
