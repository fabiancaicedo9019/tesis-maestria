\section{Justificación del trabajo de grado}
La creciente necesidad de mejorar y agilizar los procesos en el sector de Seguridad y Salud en el Trabajo ha impulsado la búsqueda de herramientas tecnológicas avanzadas que complementen y potencien las soluciones actuales. Entre estas herramientas, la Inteligencia Artificial (IA) se ha destacado, ofreciendo beneficios significativos en áreas como la automatización y el análisis predictivo.

Organizaciones como la Corporación Talentum reconocen la importancia de integrar tecnologías de IA en sus productos existentes destinados a la seguridad y salud laboral. Esta adaptación no solo responde a un deseo de optimizar procesos y aumentar la eficiencia, sino también a una estrategia para mantenerse competitivos frente a otras empresas que ya están aprovechando la IA en sus soluciones.

A pesar de los potenciales beneficios, la incorporación de IA en este sector presenta varios desafíos. Asegurar la precisión de los datos, garantizar la confiabilidad del análisis y la adaptabilidad de los sistemas existentes son retos prominentes. Adicionalmente, para la Corporación Talentum, se añade el desafío de implementar estas tecnologías tanto en productos establecidos como en proyectos gubernamentales en curso.

La integración de componentes de IA en soluciones software para la Seguridad y Salud en el Trabajo va más allá de una tendencia; se ha convertido en una necesidad palpable. Mediante esta integración, se pretende potenciar el análisis de condiciones laborales, proporcionando recomendaciones más precisas y sugiriendo acciones concretas para mejorar la seguridad de los trabajadores.

En este marco, la propuesta de investigación de maestría adquiere un valor fundamental. El proyecto justifica aún más su relevancia al buscar enfrentar y superar estos obstáculos, con el objetivo de brindar soluciones efectivas y modernas mediante la incorporación de IA en productos y en ejecuciones de proyectos gubernamentales. Así, la investigación no solo beneficia a la Corporación Talentum, sino que también establece principios y marcos que pueden influir en el sector de seguridad y salud laboral en su conjunto.
