\section{Definición del problema}

\subsection{Planteamiento del problema}
En la era digital contemporánea, la Inteligencia Artificial (IA) ha transformado rápidamente múltiples sectores, se puede decir que desde la atención médica y la manufactura hasta el entretenimiento y la logística. Su virtud para analizar gran cantidad o volúmenes de datos, hacer predicciones precisas y automatizar tareas complejas ha demostrado ser un activo invaluable. Dentro del campo del desarrollo y la arquitectura de software, la integración de la IA presenta oportunidades únicas para mejorar la eficiencia, la funcionalidad y la experiencia del usuario.

No obstante, mientras muchas organizaciones están ansiosas por adoptar y beneficiarse de las capacidades de la IA, enfrentan desafíos significativos. Uno de los principales desafíos es la falta de un conjunto estandarizado de mejores prácticas que guíen la integración de soluciones de IA en arquitecturas de software preexistentes \citep{Wang2016ImplementingOutlook}. Esta ausencia puede dar lugar a incompatibilidades, redundancias, ineficiencias y, en el peor de los casos, a fallas en el sistema.

La integración eficiente y efectiva de productos basados en IA en sistemas existentes es esencial para mantener la relevancia en un mercado en constante cambio y para aprovechar al máximo los beneficios o ventajas que ofrece la IA. Esta integración no solo implica la adaptación técnica, sino también la consideración de cómo los productos de IA pueden añadir valor real a los usuarios y a la organización \citep{Cui2022ConstructionIntelligence}. Sin una orientación clara y prácticas estandarizadas, las organizaciones pueden encontrarse con sistemas sobre complicados, costosos y que no cumplen con las expectativas.

Por otra parte, aunque la literatura ha abordado ampliamente las capacidades y aplicaciones de la IA, existe una notoria falta de investigación exhaustiva sobre la estrategia de integración de estas soluciones en el mercado tecnológico \citep{Cui2022ConstructionIntelligence}. Esta laguna en la investigación puede dificultar a las organizaciones la toma de decisiones informadas sobre cómo abordar la integración de IA.

Adicionalmente, con la rápida evolución del mercado y la tecnología, las organizaciones se ven presionadas a adaptarse rápidamente a las necesidades cambiantes \citep{Wang2016ImplementingOutlook}. Sin un marco de referencia sólido, esta adaptación puede volverse reactiva en lugar de proactiva, lo que puede llevar a soluciones temporales o inadecuadas.

En el contexto colombiano, la Corporación Talentum ha forjado una reputación sólida como ejecutora de proyectos gubernamentales, tanto tecnológicos como no tecnológicos. Estos proyectos, que buscan generar un alto impacto en distintos sectores gubernamentales, ponen un énfasis particular en la seguridad y salud en el trabajo, con el objetivo primordial de prevenir lesiones y enfermedades relacionadas con las condiciones laborales. Al mejorar las condiciones y el ambiente de trabajo, así como promover el bienestar físico, mental y social de los empleados, la corporación refleja su compromiso con la salud laboral en el país.

A lo largo de su trayectoria, la Corporación Talentum ha reconocido la trascendencia de la Inteligencia Artificial en la configuración de soluciones de vanguardia. Sin embargo, enfrenta desafíos considerables en su intento de incorporar productos de IA, acentuando la sensación de quedar rezagados frente a otras entidades que ya han adoptado estas tecnologías como parte integral de sus ofertas. Esta situación no solo se traduce en una potencial desventaja competitiva, sino que también refleja un desaprovechamiento de oportunidades para aportar valor innovador en el sector de seguridad y salud en el trabajo. La falta de un marco de trabajo adecuado para la integración de la Inteligencia Artificial ha supuesto un desafío para la Corporación Talentum, lo cual podría influir negativamente en su posicionamiento como ejecutora líder de proyectos gubernamentales en Colombia.

Por lo tanto, se hace imperativo investigar y desarrollar un marco de trabajo que proporcione un conjunto de directrices base y mejores prácticas para la integración de productos basados en IA en arquitecturas de software existentes. Esta necesidad no solo es esencial para maximizar los beneficios de la IA, sino también para garantizar la viabilidad, la escalabilidad y la relevancia a largo plazo de las soluciones tecnológicas cada vez más orientado hacia la IA.

\subsection{Formulación del problema}

Ante el emergente auge de la inteligencia artificial y su profunda influencia en el ámbito del desarrollo de software, las organizaciones enfrentan la necesidad imperativa de integrar y adaptarse a estas tecnologías para mantenerse competitivas en el mercado. Específicamente, la Corporación Talentum, que opera en el sector de seguridad y salud en el trabajo en Colombia, identifica la potencialidad y el valor agregado que la IA puede ofrecer a sus soluciones. Sin embargo, debido a la falta de directrices claras y una estrategia definida, enfrenta dificultades en la transición hacia un entorno tecnológico más avanzado y orientado a la IA.

Dado el panorama actual de las empresas que desarrollan software y teniendo en cuenta las particularidades y objetivos de la Corporación Talentum, surgen las siguientes interrogantes enfocadas en el sector de seguridad y salud en el trabajo:
\begin{itemize}
    \item ¿Cómo pueden las soluciones orientadas al sector de seguridad y salud en el trabajo incorporar efectivamente componentes de inteligencia artificial en los requerimientos funcionales del software?
    \item ¿Qué buenas prácticas, estándares y componentes arquitectónicos se deben adoptar para facilitar la integración de productos basados en IA en arquitecturas de software preexistentes?
    \item ¿De qué manera se asegura que la integración de IA en las soluciones de software no solo aporte beneficios técnicos y funcionales, sino que también respete los principios éticos del sector de seguridad y salud en el trabajo?
    \item ¿Qué consideraciones arquitectónicas son necesarias para gestionar adecuadamente los prompts de Inteligencia Artificial, asegurando una comunicación eficiente y precisa con sistemas preexistentes?
\end{itemize}

Con estas preguntas, se busca abordar la problemática central relacionada con la integración eficiente de la IA en las soluciones de software de la Corporación Talentum en el ámbito de la seguridad y salud en el trabajo. El objetivo es identificar áreas clave de desafío y establecer una estructura que guíe la investigación y desarrollo de directrices y mejores prácticas para este proyecto de maestría.
