\section{Definición del problema}
\label{sec:problema}

\subsection{Planteamiento del problema}
En la era digital actual, la Inteligencia Artificial (IA) ha transformado rápidamente múltiples sectores, se puede decir que desde la atención médica y la manufactura hasta el entretenimiento y la logística. Su virtud para analizar grandes volúmenes de datos, hacer predicciones precisas y automatizar tareas complejas ha demostrado ser un activo invaluable. 

En el \'area del Desarrollo de Software, la integración de la IA presenta oportunidades únicas para mejorar la eficiencia, la funcionalidad y la experiencia del usuario entre otros aspectos. De ah\'i que, muchas organizaciones están ansiosas por adoptar y beneficiarse de la integraci\'on de las capacidades de la IA en sus productos. La integración eficiente y efectiva de productos basados en IA en sistemas existentes es esencial para mantener la relevancia en un mercado en constante cambio y para aprovechar al máximo los beneficios o ventajas que ofrece la IA. Esta integración no solo implica la adaptación técnica, sino también la consideración de cómo los productos de IA pueden añadir valor real a los usuarios y a la organización \citep{Cui2022ConstructionIntelligence}. 

Sin embargo, estas mismas organizaciones tambi\'en deben enfrentarse a diferentes desafíos como la falta de un conjunto estandarizado de buenas prácticas que gu\'ien la integraci\'on de soluciones de IA en arquitecturas de software preexistentes \citep{Wang2016ImplementingOutlook}. Esta ausencia puede dar lugar a incompatibilidades, redundancias, ineficiencias y, en el peor de los casos, a fallas en el sistema. Sin una orientación clara y prácticas estandarizadas, las organizaciones pueden encontrarse con sistemas sobre complicados, costosos y que no cumplen con las expectativas. 

Aunque la literatura ha abordado ampliamente las capacidades y aplicaciones de la IA, existe una notoria falta de investigación exhaustiva sobre la estrategia de integración de estas soluciones en el mercado tecnológico \citep{Cui2022ConstructionIntelligence}. Esta laguna en la investigación puede dificultar a las organizaciones la toma de decisiones informadas sobre cómo abordar la integración de IA.

%Adicionalmente, con la rápida evolución del mercado y la tecnología, las organizaciones se ven presionadas a adaptarse rápidamente a las necesidades cambiantes \citep{Wang2016ImplementingOutlook}. Sin un marco de referencia sólido, esta adaptación puede volverse reactiva en lugar de proactiva, lo que puede llevar a soluciones temporales o inadecuadas.

En el contexto colombiano, la Corporación Talentum ha consolidado una sólida reputación como ejecutora de proyectos gubernamentales, tanto tecnológicos como no tecnológicos. Estos proyectos, destinados a generar un alto impacto en diferentes sectores gubernamentales, resaltan la importancia de la Seguridad y Salud en el Trabajo. El objetivo primordial de la corporación es mejorar sus procesos de Seguridad y Salud en el Trabajo. No obstante, la Corporación Talentum aspira a ejecutar proyectos gubernamentales con productos propios que incorporen Inteligencia Artificial. La intención es respaldar y ofrecer mejoras en los procesos de estas entidades, creyendo que, al hacerlo, indirectamente contribuirán a mejorar las condiciones y el ambiente laboral. Al promover el bienestar de los empleados a través de estas iniciativas, la corporación manifiesta su compromiso con la salud laboral en Colombia.

A lo largo de su trayectoria, la Corporación Talentum ha reconocido la trascendencia de la Inteligencia Artificial en la configuración de soluciones de vanguardia. Sin embargo, enfrenta desafíos considerables en su intento de incorporar productos de IA, acentuando la sensación de quedar rezagados frente a otras entidades que ya han adoptado estas tecnologías como parte integral de sus ofertas. Esta situación no solo se traduce en una potencial desventaja competitiva, sino que también refleja un desaprovechamiento de oportunidades para aportar valor innovador en el sector de Seguridad y Salud en el Trabajo. 

La falta de un conjunto estandarizado de buenas pr\'acticas para la integración de la Inteligencia Artificial, enmarcadas en un marco de trabajo, ha supuesto un desafío para la Corporación Talentum, lo cual podría influir negativamente en su posicionamiento como ejecutora líder de proyectos gubernamentales en Colombia. Por lo tanto, se hace imperativo investigar y desarrollar un marco de trabajo que proporcione un conjunto de directrices base y mejores prácticas para la integración de productos basados en IA en arquitecturas de software existentes. Esta necesidad no solo es esencial para maximizar los beneficios de la IA, sino también para garantizar la viabilidad, la escalabilidad y la relevancia a largo plazo de las soluciones tecnológicas cada vez más orientado hacia la IA.

\subsection{Formulación del problema}

%En la actualidad, la influencia creciente de la inteligencia artificial (IA) en el desarrollo de software exige que las organizaciones, incluida la Corporación Talentum en Colombia, adapten sus estrategias para mantener su competitividad. A pesar de reconocer el potencial de la IA en el sector de seguridad y salud en el trabajo, la corporación enfrenta desafíos en su incorporación debido a la falta de directrices claras y una estrategia bien definida.
Dado el contexto anterior y teniendo en cuenta las necesidades actuales de la Corporación Talentum para integrar soluciones de IA en el sector de la Salud y Seguridad en el Trabajo, surgen las siguientes interrogantes:
\begin{itemize}
    \item ¿Cómo pueden las soluciones orientadas al sector de Seguridad y Salud en el Trabajo incorporar efectivamente componentes de inteligencia artificial en los requerimientos funcionales del software?
    \item ¿Qué buenas prácticas, estándares y componentes arquitectónicos se deben adoptar para facilitar la integración de productos basados en IA en arquitecturas de software preexistentes?
    \item ¿Cuáles son los componentes potenciales de software y cuáles son sus responsabilidades para facilitar o mejorar la integración de la Inteligencia Artificial en las soluciones de software, de manera que se contribuya a otorgar beneficios técnicos y funcionales?
    \item ¿Qué consideraciones arquitectónicas son necesarias para gestionar adecuadamente los prompts de Inteligencia Artificial, asegurando una comunicación eficiente y precisa con sistemas preexistentes?
\end{itemize}

%Con estas preguntas, se busca abordar la problemática central relacionada con la integración eficiente de la IA en las soluciones de software de la Corporación Talentum en el ámbito de la seguridad y salud en el trabajo. El objetivo es identificar áreas clave de desafío y establecer una estructura que guíe la investigación y desarrollo de directrices y mejores prácticas para este proyecto de maestría.