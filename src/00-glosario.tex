\section{Glosario de Términos}
\begin{description}
    \item [Inteligencia Artificial (IA):] Se refiere a la simulación de procesos de inteligencia humana por máquinas, especialmente sistemas informáticos. Estos procesos incluyen el aprendizaje (adquisición de información y reglas para usar la información), el razonamiento (usar reglas para llegar a conclusiones aproximadas o definitivas) y la auto-corrección.
    
    \item [Marco de Trabajo (Framework):] Es una plataforma estandarizada y, a menudo, un conjunto de herramientas y bibliotecas que facilitan el desarrollo y la gestión de aplicaciones y sistemas de software. Los marcos de trabajo proporcionan una base sobre la cual se pueden desarrollar aplicaciones, asegurando consistencia, eficiencia y a menudo incorporando prácticas recomendadas.

    \item [Prototipo Funcional:] En el desarrollo de software, se refiere a una versión inicial o modelo de un programa que tiene la funcionalidad esencial para demostrar un concepto o proceso. Se utiliza para probar y refinar características antes de desarrollar una versión completa o final del software.
    
    \item [Arquitectura de Software:] Se refiere a la estructura y diseño de un sistema de software, incluyendo los componentes del sistema, las relaciones entre esos componentes y las interfaces mediante las cuales interactúan. La arquitectura de software sirve como un plan o esquema que describe cómo se integran y funcionan juntas las diferentes partes de un sistema.

    \item [Requerimientos Funcionales:] Son declaraciones detalladas de las capacidades que debe tener un sistema, las interacciones que debe soportar y las actividades que debe poder realizar. Específicamente, describen lo que hace el sistema, como procesar datos, interactuar con el usuario u operar con otros sistemas.

    \item [Seguridad y Salud en el Trabajo:] Se refiere al conjunto de disciplinas y medidas que buscan proteger y mejorar el bienestar físico, mental y social de los trabajadores en sus lugares de trabajo. Esta área se enfoca en anticipar, reconocer, evaluar y controlar aquellos factores en el ambiente laboral que pueden causar enfermedad o lesiones, y afectar el bienestar de los trabajadores y sus comunidades.

    \item [Prompt:] En el contexto de programación y sistemas informáticos, un prompt es una secuencia de caracteres que se utiliza en una interfaz de usuario para indicar la disposición para recibir entradas del usuario. También puede referirse a la invitación visual en interfaces de línea de comandos que indica al usuario que el sistema está listo para recibir comandos. En el contexto de modelos de lenguaje como OpenAI, un prompt es la entrada dada al modelo para generar una respuesta o continuación.
\end{description}


\end{description}