\thispagestyle{empty}
\begin{center}
    \Large{Ficha Resumen \\ Anteproyecto de Trabajo de Grado}
\end{center}

\textbf{Marco de trabajo para la integración de herramientas de IA en los productos de software para el área de Seguridad y Salud en el Trabajo desarrollados por la Corporación Talentum}
\begin{enumerate}
    \item Área de trabajo: Ingeniería y tecnología
    \item Tipo de proyecto: Aplicado
    \item Estudiante: Fabián Andrés Caicedo Cuellar
    \item Correo electrónico: fabiancaicedo@javerianacali.edu.co
    \item Dirección y teléfono: Carrera 23C \#10-02 / 3176367317
    \item Director: Oscar Orlando Ceballos Argote, Ph.D.
    \item Vinculación del director: Hora C\'atedra
    \item Correo electrónico del director: oscar.ceballos@javerianacali.edu.co
    %\item Co-Director (Si aplica):
    \item Grupo o empresa que lo avala (Si aplica): Corporación Talentum
    % \item Otros grupos o empresas:
    \item Palabras clave(al menos 5): Inteligencia Artificial (IA), Seguridad y Salud en el Trabajo
(SST), Ingeniería de Software (IS).
    \item Fecha de inicio: 1 de Enero de 2024
    \item Duración estimada (en meses): 6 meses
    \item Resumen:  La propuesta de investigación presentada propone un marco de trabajo integral y detallado para la incorporación efectiva de componentes basados en Inteligencia Artificial (IA) en arquitecturas de software ya existentes, poniendo especial atención en el ámbito de la seguridad y salud laboral. Mediante un análisis y revisión de la literatura existente, se examina la problemática asociada a la integración de la IA en sistemas preexistentes, identificando así los retos técnicos, arquitectónicos y contextuales que implica la implementación de esta tecnología.

En este marco, se utiliza como caso de estudio la Corporación Talentum, una entidad prominente en la implementación de proyectos gubernamentales en Colombia. Esta corporación afronta desafíos significativos para integrar productos de IA en sus soluciones, resaltando la urgente necesidad de establecer un conjunto estandarizado de buenas prácticas y directrices.

La propuesta de investigación propone un marco de trabajo que comprende una serie de buenas prácticas, componentes arquitectónicos potenciales y criterios específicos destinados a lograr una integración efectiva de la IA. El objetivo de este marco no es únicamente facilitar la adaptación técnica, sino también maximizar los beneficios derivados de la IA, asegurando así su relevancia y sostenibilidad a largo plazo. Como parte de la propuesta, se presenta un prototipo funcional web y modular, que sirve como caso práctico para validar la aplicabilidad del marco de trabajo propuesto.

Se anticipa que los resultados de este trabajo incluyan un documento que defina el marco de trabajo y proporcione buenas prácticas para la integración de la IA, acompañado de un conjunto de diagramas UML y del modelo C4 para ilustrar los componentes arquitectónicos sugeridos. Se espera, además, que el prototipo funcional demuestre de manera tangible la utilidad y aplicabilidad del marco de trabajo en contextos reales.

Este trabajo aporta significativamente al campo de la IA y la arquitectura de software, ofreciendo una guía práctica y un conjunto de herramientas útiles para la integración de productos basados en IA en soluciones ya existentes. De igual manera, subraya la necesidad de una implementación orientada de la IA, especialmente en sectores de crítica importancia como lo es la seguridad y salud en el trabajo.

\end{enumerate}
