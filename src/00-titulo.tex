\begin{center}
\thispagestyle{empty}
\vspace*{-2.2cm}
\begin{center}
    \includegraphics[width=9cm]{pujlogo}~\\[1cm]
\end{center}
\textbf{\huge
Marco de trabajo para la integración de herramientas de IA en los productos de software para el área de seguridad y salud en el trabajo desarrollados por la Corporación Talentum
}\\[1.2cm]
% Marco de trabajo para incorporación de buenas prácticas y directrices para la integración de nuevos productos con Inteligencia Artificial en las arquitecturas existentes en la Corporación Talentum

% \todo[inline]{Marco de trabajo para incorporación de buenas prácticas y directrices para la integración de nuevos productos con Inteligencia Artificial en las arquitecturas existentes en la Corporación Talentum}
% T\'{\i}tulo de la tesis  o trabajo de investigación}\\[1.75cm]
\Large\textbf{Fabian Andres Caicedo Cuellar}\\[1.2cm]
\small Anteproyecto presentada(o) como requisito parcial para optar al
t\'{\i}tulo de:\\
\textbf{Magister en Ingenier\'{\i}a de Software}\\[1.3cm]
Director(a):\\
Ph.D. Ceballos Argote, Oscar Orlando\\[1.3cm]

Pontificia Universidad Javeriana Cali\\
Facultad de Ingeniería\\
Departamento de Electrónica y Ciencias de la Computación\\
Cali, Colombia\\
\today\\
\end{center}
