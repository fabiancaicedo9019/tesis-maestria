\section{Metodología de la investigación}
\label{sec:metodologia}

Este capítulo describe la metodología de investigación adoptada para guiar la integración de la Inteligencia Artificial en soluciones de software para el sector de la Seguridad y Salud en el Trabajo. Se articula en varias fases, cada una con actividades específicas orientadas a cumplir los objetivos del proyecto.

\subsection{Fase 1: Recopilación de buenas prácticas}
La primera fase del proyecto se concentra en la identificación de las mejores prácticas para la incorporación de la Inteligencia Artificial en el software, con un enfoque particular en la gestión de los recursos computacionales, la selección de herramientas y librerías, y la elección de modelos de IA adecuados para el sector de la Seguridad y Salud en el Trabajo.

\subsubsection{Actividad 1.1: Revisión teórica de literatura existente}
Se llevará a cabo una investigación de la literatura existente para recopilar estrategias en la integración de la IA. Esta actividad implica un análisis de documentos académicos, reportes técnicos y estudios de caso que ilustren buenas prácticas en la gestión de recursos computacionales, la optimización de herramientas y librerías, y la implementación de modelos de IA.

\subsubsection{Actividad 1.2: Consolidación de prácticas para la Integración de IA}
Posterior a la revisión teórica, se consolidará un compendio de prácticas, sirviendo de referencia para la arquitectura del software y la selección de componentes de Inteligencia Artificial. Esta compilación ayudara como guía o referencia en las decisiones de las etapas de diseño y desarrollo inicial, permitiendo un lineamiento para la integración de la IA en el ámbito de SST.





\subsection{Fase 2: Análisis de Componentes Arquitectónicos}
Esta fase se enfoca en el estudio de los componentes arquitectónicos que permitirán una integración de la Inteligencia Artificial en el ámbito del software. La atención se centra en tres estilos arquitectónicos fundamentales: monolítico, microservicios y basado en servicios.

\subsubsection{Actividad 2.1: Revisión de elementos arquitectónicos}
Se procederá a identificar y analizar elementos arquitectónicos fundamentales para la integración de la Inteligencia Artificial. Se explorarán diversas fuentes para resaltar las características y beneficios de estos elementos, estableciendo un fundamento para la futura incorporación de componentes de IA en sistemas de software, con especial énfasis en su aplicabilidad en SST.

\subsubsection{Actividad 2.2: Categorización de elementos arquitectónicos}
Esta actividad se centra en la categorización de los componentes arquitectónicos seleccionados. El objetivo es desarrollar una caracterización que facilite la integración armoniosa de dichos elementos en la fase de diseño de software, promoviendo la incorporación sinérgica de soluciones de Inteligencia Artificial en plataformas especializadas en SST.





\subsection{Fase 3: Definición de Contexto en SST}
Esta fase se dedica a especificar teóricamente el contexto de la Seguridad y Salud en el Trabajo (SST) para ajustar de manera precisa la Inteligencia Artificial a los requerimientos priorizados del sector en la Corporación Talentum.

\subsubsection{Actividad 3.1: Análisis Teórico de Contexto para la IA en SST}
Se llevará a cabo un estudio teórico para entender y determinar los contextos dentro de los cuales la Inteligencia Artificial debería operar para mejorar la precisión y relevancia de sus respuestas en el ámbito de la SST. Se analizarán los factores y variables que influyen en la interacción con sistemas de IA, con el objetivo de que las respuestas generadas correspondan a las particularidades y requerimientos del sector.




\subsection{Fase 4: Adaptaciones Arquitectónicas}
Esta fase se centra en la revisión de la literatura sobre estilos arquitectónicos como monolíticos, microservicios y basados en servicios, con el fin de recopilar información pertinente que guíe las adaptaciones necesarias en las arquitecturas de software existentes para la integración de la IA.

\subsubsection{Actividad 4.1: Evaluación de cambios arquitectónicos}
Se llevará a cabo una evaluación de las adaptaciones arquitectónicas que resulten más pertinentes para sistemas preexistentes, tomando como referencia los componentes arquitectónicos identificados en la fase anterior. Esta revisión tiene como objetivo determinar las modificaciones necesarias que permitan la incorporación de la Inteligencia Artificial.

\subsubsection{Actividad 4.2: Revisión teórica sobre estilos de arquitectura de integración.}
Se profundizará en la revisión de literatura, publicaciones y documentos técnicos sobre arquitectura de integración. Se estudiarán los estilos y tácticas más actuales y su aplicabilidad al contexto del proyecto, dentro del marco del proyecto se revisarán los estilos monolíticos, microservicios y basado en servicios.





\subsection{Fase 5: Desarrollo y validación del prototipo}
Esta fase se concentra en la implementación y posterior verificación de un prototipo que ejemplifique la aplicación práctica del marco de trabajo desarrollado. Este prototipo, será construido con la colaboración de la Corporación Talentum para el levantamiento de requerimientos, funcionará como un caso de prueba para comprobar el marco propuesto en atender las necesidades específicas del sector de SST.

\subsubsection{Actividad 5.1: Análisis y recolección de requerimientos}
Se efectuará una comprensión de las necesidades y expectativas de la Corporación Talentum. Esta actividad se enfocará en recoger y analizar los requisitos específicos para el prototipo de software, garantizando que la solución final esté perfectamente alineada con las demandas del sector de SST.

\subsubsection{Actividad 5.2: Priorización de requerimientos}
A través de talleres colaborativos, se priorizarán los requerimientos del prototipo. Esta priorización es para desarrollar una hoja de ruta estratégica para el proyecto y para prever retos potenciales en la fase de implementación.

\subsubsection{Actividad 5.3: Diseño y desarrollo de un prototipo aplicando el marco de trabajo}
Con los requerimientos ya establecidos, se dará inicio al proceso de diseño y desarrollo del prototipo. Esta fase aplicará el marco de trabajo diseñado para construir un caso de estudio que demuestre la aplicabilidad del marco en el cumplimiento de los objetivos del proyecto, particularmente en la integración de IA en soluciones de SST.

\subsubsection{Actividad 5.4: Evaluación del uso del marco de trabajo en el prototipo}
Finalmente, se evaluará cómo la aplicación del marco de trabajo ha influenciado positivamente el desarrollo del prototipo, especialmente en lo que respecta a la incorporación de la IA. Se verificará que el marco de trabajo no solo ha facilitado el proceso de desarrollo, sino que también ha guiado la atención a los requerimientos de IA, garantizando resultados alineados con las expectativas del proyecto.





