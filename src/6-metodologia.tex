\section{Metodología de la investigación}

Para garantizar una incorporación efectiva de componentes de Inteligencia Artificial en soluciones software dirigidas al sector de seguridad y salud en el trabajo, se establece una serie de fases y actividades sistemáticas. Estas fases están diseñadas para atender los requerimientos específicos y cumplir con los objetivos propuestos para el proyecto.

\subsection{Fase 1: Análisis e identificación de Requerimientos}
Antes de iniciar, es esencial comprender a fondo las necesidades y expectativas. En el ámbito de las soluciones software, es crucial identificar los requerimientos en este caso de la Corporación Talentum. En esta fase, se dedica un esfuerzo para identificar las especificaciones relevantes. El enfoque aquí no solo es recolectar la información, sino también entender el contexto detrás de cada requerimiento para que el prototipo final pueda satisfacer de manera óptima las demandas del sector.

\subsubsection{Actividad 1.1: Obtención de requerimientos y características mínimas del prototipo.}
Se organizará y conducirá una serie de reuniones estructuradas con profesionales de la Corporación Talentum que tienen experiencia en proyectos relacionados con la seguridad y salud en el trabajo. A través de estas reuniones, se buscará recopilar e identificar las especificaciones detalladas, características y funciones que el prototipo debe incluir. El objetivo es definir un conjunto de requerimientos que respondan a las necesidades actuales del sector.

\subsubsection{Actividad 1.2: Evaluación de la prioridad del primer requerimiento del prototipo en Corporación Talentum.}
Mediante sesiones de trabajo interactivas, se evaluarán las necesidades y prioridades que la Corporación Talentum considera más urgentes. Esta evaluación permitirá determinar cuál requerimiento deberá abordarse primero y establecer una hoja de ruta para el desarrollo del prototipo. Además, se identificarán posibles desafíos o limitaciones asociados con la implementación de este requerimiento.

\subsubsection{Actividad 1.3: Documentación de la arquitectura de software y componentes necesarios.}
Con base en los requerimientos identificados, se procederá a diseñar una arquitectura de software coherente que permita una integración eficiente de las diferentes funcionalidades. Esta arquitectura se documentará en detalle, incluyendo los componentes y módulos necesarios, sus interrelaciones, y se asegurará que sea modular, interoperable y escalable para adaptarse a futuras necesidades.

\subsection{Fase 2: Diseño y Desarrollo del Prototipo}
Una vez establecidos los requerimientos, el siguiente paso es el desarrollo del proyecto. En esta fase, se traza la ruta para convertir las especificaciones teóricas en un prototipo tangible. Es un proceso iterativo, donde el diseño y el desarrollo se llevan a cabo simultáneamente, permitiendo adaptaciones basadas en descubrimientos realizados durante el proceso. La colaboración y la comunicación constante con los stakeholders, en este caso la Corporación Talentum, es esencial para garantizar que el producto final esté alineado con las visiones y necesidades iniciales.

\subsubsection{Actividad 2.1: Selección de tecnología para el desarrollo del prototipo.}
Se llevará a cabo un análisis exhaustivo sobre las tecnologías y herramientas actuales en el mercado que sean relevantes para el proyecto. Esta investigación contemplará aspectos como compatibilidad, escalabilidad, soporte técnico y costos asociados. Tras este análisis, se decidirá sobre las tecnologías que ofrecen la mayor adaptabilidad y rendimiento para el prototipo.

\subsubsection{Actividad 2.2: Inicio del desarrollo del prototipo basado en los requerimientos de Corporación Talentum.}
Se iniciará el desarrollo del prototipo, respetando las especificaciones y requerimientos recabados. Durante esta fase, se mantendrá una comunicación constante con la Corporación Talentum, asegurando que el desarrollo esté alineado con sus expectativas y necesidades. Se realizarán iteraciones y ajustes según el feedback recibido.

\subsection{Fase 3: Definición del Marco de Trabajo y Componentes Tecnológicos}
Con el diseño y desarrollo en marcha, es esencial establecer un marco de trabajo que guíe las integraciones tecnológicas y facilite la escalabilidad del proyecto. Esta fase se concentra en definir y documentar el marco de trabajo con la arquitectura de software y las herramientas tecnológicas que se utilizarán. La investigación teórica en esta etapa ayuda a garantizar que las decisiones tomadas estén fundamentadas en prácticas probadas y estén alineadas con las tendencias actuales de la industria.

\subsubsection{Actividad 3.1: Revisión teórica sobre estilos de arquitectura de integración.}
Se profundizará en la revisión de literatura, publicaciones y documentos técnicos sobre arquitectura de integración. Se estudiarán los estilos y tácticas más actuales y su aplicabilidad al contexto del proyecto.

\subsubsection{Actividad 3.2: Revisión teórica sobre patrones de integración.}
Se consultará literatura especializada y se analizarán estudios de caso para obtener un panorama claro sobre los patrones de integración que han demostrado ser efectivos en proyectos similares. Se buscará identificar y documentar patrones que puedan ser replicados o adaptados para el proyecto en cuestión.

\subsubsection{Actividad 3.3: Revisión teórica de construcción de prompts.}
Se llevará a cabo una investigación teórica sobre la construcción y especificación de prompts para garantizar que los mismos sean claros, relevantes y alineados con las necesidades del usuario. Se investigará cómo estos prompts pueden ser estructurados para abordar de manera efectiva las necesidades y expectativas de la Corporación Talentum.

\subsection{Fase 4: Validación y Pruebas}
Esta fase se dedica a validar que el prototipo no solo cumpla con las especificaciones técnicas, sino que también satisfaga a los usuarios. A través de una serie de pruebas, se identificarán y rectificarán las áreas problemáticas, garantizando que el prototipo final cumpla con las definiciones previas.

\subsubsection{Actividad 4.1: Pruebas unitarias.}
Se realizarán pruebas unitarias sobre los componentes de software. Esto garantizará que cada función o módulo opere correctamente y cumpla con las especificaciones técnicas previamente definidas. Además, se buscará identificar y corregir cualquier fallo o incoherencia en esta etapa temprana del proceso de prueba.

\subsubsection{Actividad 4.2: Pruebas de Aceptación.}
Se organizarán sesiones en las que representantes de la Corporación Talentum interactuarán con el prototipo. Estas pruebas de aceptación permitirán evaluar funcionalidad. A través de estas interacciones, se socializará al equipo de la Corporación Talentum todo el prototipo.







% La metodología debe reflejar la estructura lógica del proceso de investigación. Esta sección define y explica la selección de la estrategia adoptada para responder al problema planteado y además explica el cómo va a realizar la investiga. 

% La metodología indica cómo será el proceso desde la recolección de los datos, la organización, sistematización, y análisis de la información, hasta la forma como se van a interpretar y presentar los resultados.   Si bien esto puede cambiar en la realización del proyecto una metodología concreta permite tener una guía para la elaboración del proyecto. 

% La metodología definida debe reflejar la articulación entre los objetivos, el proyecto y los procedimientos para cumplir dichos objetivos. 

