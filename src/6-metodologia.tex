\section{Metodología de la investigación}
\label{sec:metodologia}

Este capítulo describe la metodología de investigación adoptada para guiar la integración de la Inteligencia Artificial en soluciones de software para el sector de la Seguridad y Salud en el Trabajo. Se articula en varias fases, cada una con actividades específicas orientadas a cumplir los objetivos del proyecto.

\subsection{Fase 1: Recopilación de buenas prácticas}
La primera fase del proyecto se concentra en la identificación de las mejores prácticas para la incorporación de la Inteligencia Artificial en el software, con un enfoque particular en la gestión de los recursos computacionales, la selección de herramientas y librerías, y la elección de modelos de IA adecuados para el sector de la Seguridad y Salud en el Trabajo.

\subsubsection{Actividad 1.1: Revisión teórica de literatura existente}
Se llevará a cabo una investigación de la literatura existente para recopilar estrategias en la integración de la IA. Esta actividad implica un análisis de documentos académicos, reportes técnicos y estudios de caso que ilustren buenas prácticas en la gestión de recursos computacionales, la optimización de herramientas y librerías, y la implementación de modelos de IA.

\subsubsection{Actividad 1.2: Consolidación de prácticas para la Integración de IA}
Posterior a la revisión teórica, se consolidará un compendio de prácticas, sirviendo de referencia para la arquitectura del software y la selección de componentes de Inteligencia Artificial. Esta compilación ayudara como guía o referencia en las decisiones de las etapas de diseño y desarrollo inicial, permitiendo un lineamiento para la integración de la IA en el ámbito de SST.





\subsection{Fase 2: Análisis de Componentes Arquitectónicos}
Esta fase se enfoca en el estudio de los componentes arquitectónicos que permitirán una integración de la Inteligencia Artificial en el ámbito del software. La atención se centra en tres estilos arquitectónicos fundamentales: monolítico, microservicios y basado en servicios.

\subsubsection{Actividad 2.1: Revisión de elementos arquitectónicos}
Se procederá a identificar y analizar elementos arquitectónicos fundamentales para la integración de la Inteligencia Artificial. Se explorarán diversas fuentes para resaltar las características y beneficios de estos elementos, estableciendo un fundamento para la futura incorporación de componentes de IA en sistemas de software, con especial énfasis en su aplicabilidad en SST.

\subsubsection{Actividad 2.2: Categorización de elementos arquitectónicos}
Esta actividad se centra en la categorización de los componentes arquitectónicos seleccionados. El objetivo es desarrollar una caracterización que facilite la integración armoniosa de dichos elementos en la fase de diseño de software, promoviendo la incorporación sinérgica de soluciones de Inteligencia Artificial en plataformas especializadas en SST.





\subsection{Fase 3: Definición de Contexto en SST}
Esta fase se dedica a especificar teóricamente el contexto de la Seguridad y Salud en el Trabajo (SST) para ajustar de manera precisa la Inteligencia Artificial a los requerimientos priorizados del sector en la Corporación Talentum.

\subsubsection{Actividad 3.1: Análisis Teórico de Contexto para la IA en SST}
Se llevará a cabo un estudio teórico para entender y determinar los contextos dentro de los cuales la Inteligencia Artificial debería operar para mejorar la precisión y relevancia de sus respuestas en el ámbito de la SST. Se analizarán los factores y variables que influyen en la interacción con sistemas de IA, con el objetivo de que las respuestas generadas correspondan a las particularidades y requerimientos del sector.




\subsection{Fase 4: Adaptaciones Arquitectónicas}
Esta fase se centra en la revisión de la literatura sobre estilos arquitectónicos como monolíticos, microservicios y basados en servicios, con el fin de recopilar información pertinente que guíe las adaptaciones necesarias en las arquitecturas de software existentes para la integración de la IA.

\subsubsection{Actividad 4.1: Evaluación de cambios arquitectónicos}
Se llevará a cabo una evaluación de las adaptaciones arquitectónicas que resulten más pertinentes para sistemas preexistentes, tomando como referencia los componentes arquitectónicos identificados en la fase anterior. Esta revisión tiene como objetivo determinar las modificaciones necesarias que permitan la incorporación de la Inteligencia Artificial.

\subsubsection{Actividad 4.2: Revisión teórica sobre estilos de arquitectura de integración.}
Se profundizará en la revisión de literatura, publicaciones y documentos técnicos sobre arquitectura de integración. Se estudiarán los estilos y tácticas más actuales y su aplicabilidad al contexto del proyecto, dentro del marco del proyecto se revisarán los estilos monolíticos, microservicios y basado en servicios.





\subsection{Fase 5: Desarrollo y validación del prototipo}
Esta fase se concentra en la implementación y posterior verificación de un prototipo que ejemplifique la aplicación práctica del marco de trabajo desarrollado. Este prototipo, será construido con la colaboración de la Corporación Talentum para el levantamiento de requerimientos, funcionará como un caso de prueba para comprobar el marco propuesto en atender las necesidades específicas del sector de SST.

\subsubsection{Actividad 5.1: Análisis y recolección de requerimientos}
Se efectuará una comprensión de las necesidades y expectativas de la Corporación Talentum. Esta actividad se enfocará en recoger y analizar los requisitos específicos para el prototipo de software, garantizando que la solución final esté perfectamente alineada con las demandas del sector de SST.

\subsubsection{Actividad 5.2: Priorización de requerimientos}
A través de talleres colaborativos, se priorizarán los requerimientos del prototipo. Esta priorización es para desarrollar una hoja de ruta estratégica para el proyecto y para prever retos potenciales en la fase de implementación.

\subsubsection{Actividad 5.3: Diseño y desarrollo de un prototipo aplicando el marco de trabajo}
Con los requerimientos ya establecidos, se dará inicio al proceso de diseño y desarrollo del prototipo. Esta fase aplicará el marco de trabajo diseñado para construir un caso de estudio que demuestre la aplicabilidad del marco en el cumplimiento de los objetivos del proyecto, particularmente en la integración de IA en soluciones de SST.

\subsubsection{Actividad 5.4: Evaluación del uso del marco de trabajo en el prototipo}
Finalmente, se evaluará cómo la aplicación del marco de trabajo ha influenciado positivamente el desarrollo del prototipo, especialmente en lo que respecta a la incorporación de la IA. Se verificará que el marco de trabajo no solo ha facilitado el proceso de desarrollo, sino que también ha guiado la atención a los requerimientos de IA, garantizando resultados alineados con las expectativas del proyecto.








% Para garantizar una incorporación efectiva de componentes de Inteligencia Artificial en soluciones software dirigidas al sector de seguridad y salud en el trabajo, se establece una serie de fases y actividades sistemáticas. Estas fases están diseñadas para atender los requerimientos específicos y cumplir con los objetivos propuestos para el proyecto.

% \subsection{Fase 1: Análisis e identificación de Requerimientos}
% Antes de iniciar, es esencial comprender a fondo las necesidades y expectativas. En el ámbito de las soluciones software, es crucial identificar los requerimientos en este caso de la Corporación Talentum. En esta fase, se dedica un esfuerzo para identificar las especificaciones relevantes. El enfoque aquí no solo es recolectar la información, sino también entender el contexto detrás de cada requerimiento para que el prototipo final pueda satisfacer de manera óptima las demandas del sector.

% \subsubsection{Actividad 1.1: Obtención de requerimientos y características mínimas del prototipo.}
% Se organizará y conducirá una serie de reuniones estructuradas con profesionales de la Corporación Talentum que tienen experiencia en proyectos relacionados con la seguridad y salud en el trabajo. A través de estas reuniones, se buscará recopilar e identificar las especificaciones detalladas, características y funciones que el prototipo debe incluir. El objetivo es definir un conjunto de requerimientos que respondan a las necesidades actuales del sector.

% \subsubsection{Actividad 1.2: Evaluación de la prioridad del primer requerimiento del prototipo en Corporación Talentum.}
% Mediante sesiones de trabajo interactivas, se evaluarán las necesidades y prioridades que la Corporación Talentum considera más urgentes. Esta evaluación permitirá determinar cuál requerimiento deberá abordarse primero y establecer una hoja de ruta para el desarrollo del prototipo. Además, se identificarán posibles desafíos o limitaciones asociados con la implementación de este requerimiento.

% \subsubsection{Actividad 1.3: Documentación de la arquitectura de software y componentes necesarios.}
% Con base en los requerimientos identificados, se procederá a diseñar una arquitectura de software coherente que permita una integración eficiente de las diferentes funcionalidades. Esta arquitectura se documentará en detalle, incluyendo los componentes y módulos necesarios, sus interrelaciones, y se asegurará que sea modular, interoperable y escalable para adaptarse a futuras necesidades.

% \subsection{Fase 2: Diseño y Desarrollo del Prototipo}
% Una vez establecidos los requerimientos, el siguiente paso es el desarrollo del proyecto. En esta fase, se traza la ruta para convertir las especificaciones teóricas en un prototipo. Es un proceso iterativo, donde el diseño y el desarrollo se llevan a cabo simultáneamente, permitiendo adaptaciones basadas en descubrimientos realizados durante el proceso. La colaboración y la comunicación constante con los stakeholders, en este caso la Corporación Talentum, es esencial para garantizar que el producto final esté alineado con las visiones y necesidades iniciales.

% \subsubsection{Actividad 2.1: Selección de tecnología para el desarrollo del prototipo.}
% Se llevará a cabo un análisis exhaustivo sobre las tecnologías y herramientas actuales en el mercado que sean relevantes para el proyecto. Esta investigación contemplará aspectos como compatibilidad, escalabilidad, soporte técnico y costos asociados. Tras este análisis, se decidirá sobre las tecnologías que ofrecen la mayor adaptabilidad y rendimiento para el prototipo.

% \subsubsection{Actividad 2.2: Inicio del desarrollo del prototipo basado en los requerimientos de Corporación Talentum.}
% Se iniciará el desarrollo del prototipo, respetando las especificaciones y requerimientos recabados. Durante esta fase, se mantendrá una comunicación constante con la Corporación Talentum, asegurando que el desarrollo esté alineado con sus expectativas y necesidades. Se realizarán iteraciones y ajustes según el feedback recibido.

% \subsection{Fase 3: Definición del Marco de Trabajo y Componentes Tecnológicos}
% Con el diseño y desarrollo en marcha, es esencial establecer un marco de trabajo que guíe las integraciones tecnológicas y facilite la escalabilidad del proyecto. Esta fase se concentra en definir y documentar el marco de trabajo con la arquitectura de software y las herramientas tecnológicas que se utilizarán. La investigación teórica en esta etapa ayuda a garantizar que las decisiones tomadas estén fundamentadas en prácticas probadas y estén alineadas con las tendencias actuales de la industria.

% \subsubsection{Actividad 3.1: Revisión teórica sobre estilos de arquitectura de integración.}
% Se profundizará en la revisión de literatura, publicaciones y documentos técnicos sobre arquitectura de integración. Se estudiarán los estilos y tácticas más actuales y su aplicabilidad al contexto del proyecto.

% \subsubsection{Actividad 3.2: Revisión teórica sobre patrones de integración.}
% Se consultará literatura especializada y se analizarán estudios de caso para obtener un panorama claro sobre los patrones de integración que han demostrado ser efectivos en proyectos similares. Se buscará identificar y documentar patrones que puedan ser replicados o adaptados para el proyecto en cuestión.

% \subsubsection{Actividad 3.3: Revisión teórica de especificación de contextos para interoperar con la IA en el ámbito de seguridad y salud en el trabajo.}
% Se realizará una revisión teórica exhaustiva para comprender y definir los contextos necesarios que mejoren la precisión y relevancia de las respuestas proporcionadas por la Inteligencia Artificial (IA) en el ámbito de la seguridad y salud en el trabajo. Se enfocará en identificar y especificar los parámetros y variables que deben ser considerados al interactuar con sistemas de IA, asegurando que las respuestas generadas estén alineadas con las necesidades y expectativas de la Corporación Talentum. Además, se explorarán las mejores prácticas y estrategias para estructurar y formular prompts de manera efectiva, contribuyendo así a una interacción más eficiente y precisa con las herramientas de IA.

% \subsection{Fase 4: Validación y Pruebas}
% Esta fase se dedica a validar que el prototipo no solo cumpla con las especificaciones técnicas, sino que también satisfaga a los usuarios. A través de una serie de pruebas, se identificarán y rectificarán las áreas problemáticas, garantizando que el prototipo final cumpla con las definiciones previas.

% \subsubsection{Actividad 4.1: Pruebas unitarias.}
% Se realizarán pruebas unitarias sobre los componentes de software. Esto garantizará que cada función o módulo opere correctamente y cumpla con las especificaciones técnicas previamente definidas. Además, se buscará identificar y corregir cualquier fallo o incoherencia en esta etapa temprana del proceso de prueba.

% \subsubsection{Actividad 4.2: Pruebas de Aceptación.}
% Se organizarán sesiones en las que representantes de la Corporación Talentum interactuarán con el prototipo. Estas pruebas de aceptación permitirán evaluar funcionalidad. A través de estas interacciones, se socializará al equipo de la Corporación Talentum todo el prototipo.


