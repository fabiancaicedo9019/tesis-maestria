\section{Marco teórico de referencia y antecedentes}
\label{sec:marco}

\subsection{Bases Teóricas}

\subsubsection{Definición y Evolución de la Inteligencia Artificial (IA)}
La Inteligencia Artificial (IA) puede definirse como una rama de la ciencia de la computación que se dedica a desarrollar algoritmos, sistemas y técnicas que permiten a las máquinas aprender y realizar tareas que, hasta hace poco, solo podrían ser realizadas por seres humanos, como el reconocimiento de patrones, la toma de decisiones y la resolución de problemas complejos. En el contexto de la evaluación de la susceptibilidad a movimientos en masa, la IA ha demostrado ser una herramienta invaluable.

En este sentido, Ospina-Gutiérrez y Aristizábal (2021) presentaron una aplicación significativa de la IA en la evaluación de la susceptibilidad a movimientos en masa, específicamente en la cuenca de la quebrada La Miel, en los Andes colombianos. En su estudio, utilizaron diferentes algoritmos de aprendizaje automático para evaluar la capacidad de predicción entre varios modelos, demostrando que los modelos ensamblados tipo boosting superaron significativamente a los modelos paramétricos lineales en términos de desempeño y capacidad de predicción. Este estudio no solo destacó la eficacia de la IA en la predicción y evaluación de áreas susceptibles a movimientos en masa, sino que también enfatizó la importancia de contar con inventarios detallados de movimientos en masa y variables predictoras para el ajuste y desarrollo de modelos útiles para la toma de decisiones y la comprensión del fenómeno \citep{Ospina2021AplicacionMasa}.

\subsubsection{Componentes de Inteligencia Artificial}
Los componentes de la inteligencia artificial, particularmente en el contexto de los Modelos de Lenguaje Pre-entrenados (LLM, por sus siglas en inglés), han demostrado ser herramientas significativas en la era moderna. Una de las incorporaciones prominentes en este ámbito es el desarrollo de chatbots, los cuales hacen uso de sofisticados modelos de lenguaje como el ChatGPT para generar respuestas y mantener conversaciones fluidas con los usuarios \citep{Zamfirescu-Pereira2023WhyPrompts}.

Un aspecto central en la funcionalidad de estos sistemas es el uso de "prompts", que pueden describirse como instrucciones textuales dadas a un LLM para guiar su generación de texto. Los "prompts" pueden ser tanto simples como complejos, integrando elementos variados como preguntas, declaraciones, ejemplos, e instrucciones, que sirven para direccionar las respuestas del modelo de manera más precisa y pertinente hacia una tarea específica \citep{Zamfirescu-Pereira2023WhyPrompts}.

Aunque los "prompts" representan una herramienta vital para potenciar la calidad de las salidas generadas por los LLM, diseñar prompts efectivos puede presentar un desafío considerable, especialmente para individuos que no son expertos en el campo de la inteligencia artificial. Es esencial que los prompts sean confeccionados con un grado específico de detalle para orientar adecuadamente la generación textual, sin restringir excesivamente la capacidad creativa del modelo \citep{Zamfirescu-Pereira2023WhyPrompts}.

Asimismo, los creadores de prompts deben tener en cuenta el contexto específico y la audiencia destinataria, lo que podría requerir una comprensión profunda de la tarea en mano y del modelo de lenguaje subyacente. Este equilibrio delicado significa que diseñar interacciones efectivas con chatbots basados en LLM puede ser una tarea compleja para los no expertos en IA, representando una barrera significativa en la ingeniería de prompts efectivos para los usuarios finales \citep{Zamfirescu-Pereira2023WhyPrompts}.

 A pesar de estos desafíos, la investigación de \citet{Zamfirescu-Pereira2023WhyPrompts} sugiere que los LLM y los prompts tienen un alcance considerable en la sociedad moderna, con implicancias que van más allá de la mera interacción con chatbots. En consecuencia, es imperativo fomentar discusiones y consideraciones más profundas sobre estos componentes y cómo pueden ser integrados de manera más precisa en sistemas que tendrán un impacto palpable en la sociedad.


\subsubsection{Seguridad y salud en el trabajo}
La ``Seguridad y Salud en el Trabajo'' es un campo que abarca medidas y estrategias dirigidas a mantener y promover el bienestar físico y psicológico de los trabajadores \citep{Yaneth2021StrategiesSector,GonzalezDelgado2023AcuteStudy}. En el contexto colombiano, esta área adquiere especial relevancia dada la dinámica de varios sectores industriales, como lo son el de la salud y el de la construcción, que enfrentan desafíos específicos en términos de estrés laboral agudo y la necesidad de implementar estrategias efectivas de capacitación, respectivamente \citep{Yaneth2021StrategiesSector,GonzalezDelgado2023AcuteStudy}.

El estudio realizado por \citep{GonzalezDelgado2023AcuteStudy} arroja luz sobre la prevalencia del estrés agudo entre los trabajadores de la salud en Colombia durante el período 2017-2021. Aunque el contexto es el sector salud, este estudio proporciona una evidencia clara de la necesidad crítica de estrategias y herramientas que puedan ayudar a mitigar estos problemas de estrés, especialmente en sectores en rápida evolución, como lo es el de la tecnología, donde la integración de soluciones de inteligencia artificial podría dar lugar a desafíos similares en términos de bienestar y salud ocupacional \citep{GonzalezDelgado2023AcuteStudy}.

Simultáneamente, el estudio de \citep{Yaneth2021StrategiesSector} destaca la importancia de la formación y capacitación en el sector de la construcción, subrayando la necesidad de desarrollar herramientas y estrategias efectivas para promover la Seguridad y la Salud en el Trabajo. Aunque el estudio está centrado en el sector de la construcción, proporciona insights valiosos que podrían ser aplicables en la Corporación Talentum, a medida que se esfuerza por integrar nuevos productos con inteligencia artificial en sus estructuras existentes, garantizando así que el proceso no solo sea innovador, sino que también promueva un ambiente de trabajo seguro y saludable \citep{Yaneth2021StrategiesSector}.

En resumen, estas investigaciones ofrecen una guía valiosa para la Corporación Talentum en su búsqueda por desarrollar un marco de trabajo que no solo fomente la innovación y la integración de tecnologías avanzadas, sino que también priorice la salud y la seguridad de sus empleados, alineándose así con las mejores prácticas reconocidas en el ámbito colombiano \citep{Yaneth2021StrategiesSector,GonzalezDelgado2023AcuteStudy}.


\subsubsection{Definición de Requerimientos Funcionales}
En el ámbito de la ingeniería del software, es fundamental conceptualizar adecuadamente los términos clave que conforman la estructura de un proyecto. Dentro de este marco, se sitúan los requerimientos, una pieza esencial en el proceso de desarrollo de software, cuya adecuada definición incide directamente en la efectividad y eficacia del producto final \citep{Cipriano2023GPT-3Report,Wu2023AgileDesign}.

En primera instancia, es imperativo clarificar qué son los requerimientos en el contexto de la ingeniería del software. Los requerimientos se categorizan como especificaciones tanto funcionales como no funcionales que los programas de software deben cumplir para satisfacer las necesidades y expectativas de los usuarios y del mercado \citep{Cipriano2023GPT-3Report,Wu2023AgileDesign}. Ahondando en esto, los requerimientos funcionales hacen referencia a las funciones específicas que el software deberá ser capaz de realizar, en tanto que los requerimientos no funcionales se refieren a aspectos más abstractos, como la seguridad, la escalabilidad y la usabilidad del software \citep{Cipriano2023GPT-3Report}.

Centrándonos en los requerimientos funcionales, estos se configuran como instrucciones directas que delinean las funcionalidades que el software debe brindar. En el contexto de una tarea de programación orientada a objetos asignada a GPT-3, por ejemplo, estos requerimientos se traducen en la creación de una jerarquía de clases que simbolizan a los empleados de una empresa de TI, con funciones específicas como la calculación del salario de los empleados y la identificación de la clase jerárquica a la que pertenece cada función \citep{Cipriano2023GPT-3Report}.

En cuanto a la integración de la Inteligencia Artificial (IA) en el desarrollo de productos de software, se destaca el papel fundamental que juega en la optimización de los procesos de diseño. En particular, ChatGPT emerge como una herramienta vital para facilitar una mayor comprensión de las necesidades del usuario y de las dinámicas del mercado \citep{Wu2023AgileDesign}. \citet{Wu2023AgileDesign} profundiza en cómo ChatGPT, como tecnología avanzada de IA, puede asistir inteligentemente a los diseñadores, soportando la toma de decisiones y permitiendo una mejor respuesta a las demandas del mercado, lo que culmina en una aceleración de los ciclos de desarrollo de productos y una mejora en la competitividad del producto.

El proceso para establecer los requisitos del producto, particularmente en el contexto del diseño de Productos Mínimos Viables (MVPs), involucra una metodología que combina la investigación de mercado, la retroalimentación del usuario y el análisis de la competencia. Esta combinación se utiliza para formular una lista exhaustiva de funcionalidades, aspectos de rendimiento y otros factores críticos que deben tenerse en cuenta durante el diseño \citep{Wu2023AgileDesign}. Aquí, ChatGPT se manifiesta como un aliado crucial, facilitando la refinación y optimización de la lista de requerimientos a través de un análisis meticuloso y data-driven, lo que a su vez potencia la eficiencia y calidad del diseño \citep{Wu2023AgileDesign}.

Sin embargo, es igualmente vital considerar que la implementación de IA, como ChatGPT, en el proceso de desarrollo de software, no está exenta de desafíos. Uno de los obstáculos preponderantes radica en la necesidad de entrenar y ajustar continuamente el modelo de IA para que se adapte eficazmente a las necesidades específicas del producto. Además, surgen preocupaciones sobre la interpretabilidad del modelo y la protección de la privacidad de los datos \citep{Wu2023AgileDesign}.

En el caso particular de GPT-3 en la resolución de tareas de programación orientada a objetos, se evidenció que, aunque era capaz de cumplir con los requerimientos funcionales específicos, el código generado no siempre adhería a las mejores prácticas de diseño orientado a objetos, resultando a menudo en código que era desafiante para entender y mantener \citep{Cipriano2023GPT-3Report}. Este escenario señala una limitación significativa, sugiriendo la necesidad de investigaciones adicionales para evaluar cómo estas herramientas de IA pueden ser empleadas de manera más efectiva en entornos educativos y profesionales.

Conclusivamente, la incorporación de IA en el proceso de desarrollo de software promete una revolución significativa, proporcionando una ruta hacia la eficiencia mejorada, una calidad de diseño superior, y una innovación revolucionaria, especialmente en el ámbito del desarrollo de Productos Mínimos Viables (MVPs). Esta transformación es evidente en la utilidad del modelo de procesamiento de lenguaje natural ChatGPT, el cual facilita la comprensión profunda de las necesidades de los usuarios y las tendencias del mercado, optimizando, así, la eficiencia y la calidad del diseño de productos \citep{Wu2023AgileDesign}.

Al considerar los requerimientos funcionales, estos hacen referencia a las especificaciones que delinean las funciones que un software debe realizar. Estos son cruciales en la fase de definición de cualquier proyecto de desarrollo de software, ya que establecen las operaciones fundamentales que deben ser implementadas para satisfacer las necesidades de los usuarios finales. En el contexto de los proyectos abordados por ChatGPT, los requerimientos funcionales pueden involucrar la identificación de los usuarios objetivo y la determinación de las funcionalidades centrales que servirán para mejorar la experiencia del usuario y satisfacer las demandas del mercado \citep{Wu2023AgileDesign}.

Es imperativo que los diseñadores y desarrolladores estén conscientes de las implicancias de la integración de IA, tanto en términos de oportunidades como de desafíos. En particular, se deben tener en cuenta consideraciones clave como la interpretabilidad del modelo y la protección de la privacidad de los datos \citep{Wu2023AgileDesign}.

Por otro lado, en el contexto educativo, la herramienta de generación de lenguaje natural desarrollada por OpenAI, GPT-3, ha mostrado potencial para ayudar a resolver tareas de programación orientada a objetos. Si bien GPT-3 puede interpretar y gestionar requerimientos funcionales directos, tiene la tendencia de no proporcionar la mejor solución en términos de diseño orientado a objetos, a menudo resultando en código que puede ser difícil de entender y mantener \citep{Cipriano2023GPT-3Report}. Esto señala la necesidad de más investigaciones y adaptaciones para mejorar su utilidad en este contexto, especialmente en lo que respecta a adherirse a las mejores prácticas de diseño orientado a objetos y facilitar la creación de código que sea tanto funcional como sostenible \citep{Cipriano2023GPT-3Report}.

En vista de los hallazgos actuales, es evidente que nos encontramos en el umbral de una era de innovación y eficiencia mejorada en el diseño y desarrollo de software, con la IA desempeñando un papel crucial en este avance. No obstante, resulta fundamental profundizar en la exploración de la inteligencia artificial en campos tales como la Seguridad y Salud en el Trabajo. Es esencial evaluar cómo estas nuevas tecnologías pueden contribuir significativamente no solo en otros sectores, sino también en la mejora y garantía de la seguridad y salud ocupacional, fomentando así la integración de innovaciones tanto en productos existentes como en nuevos desarrollos en este ámbito.

\subsection{Estado del Arte}
El estado del arte que se presenta a continuación pretende evidenciar la confluencia de dos campos cruciales: la salud y bienestar de los trabajadores en su entorno laboral y la incorporación de la inteligencia artificial en este ámbito. La Seguridad y Salud en el Trabajo no solo se centra en la prevención de accidentes, sino también en la salud integral del trabajador, lo que incluye aspectos médicos y de atención sanitaria. Es por esta razón que la literatura a explorar abordará temáticas que entrelazan la atención médica con aplicaciones de IA. Estas integraciones emergen como piezas fundamentales para el desarrollo y mejoramiento de productos y sistemas destinados a salvaguardar y promover la salud de los trabajadores en sus respectivos espacios laborales. Así, el lector encontrará en este estado del arte información sobre productos de IA, así como de componentes arquitectónicos e integraciones actuales aplicadas a la salud o salud ocupacional.

\subsubsection{ChatGPT en el Ámbito Médico: Implicaciones, Potencialidades y Retos en la Atención de Salud Laboral}
El mundo ha sido testigo de cómo ChatGPT, creado por OpenAI, ha causado un profundo impacto en diversos campos, incluido el de la atención médica. En solo dos meses después de su lanzamiento, ChatGPT atrajo a 100 millones de usuarios, superando incluso a plataformas previamente populares como TikTok \citep{Kleesiek2023AnOnly}. Esto demuestra su gran influencia y su potencial en diversos sectores, incluido el de la salud laboral.

La esencia de ChatGPT reside en su tecnología subyacente: un modelo de lenguaje grande (LLM) conocido como generative pretrained transformer (GPT-3.5) entrenado con 175 mil millones de parámetros \citep{Kleesiek2023AnOnly}. Estos LLMs, originados en el procesamiento de lenguaje natural, han demostrado ser modelos fundamentales que pueden adaptarse a una amplia variedad de tareas debido a sus capacidades de aprendizaje con pocos ejemplos y transferencia de conocimiento \citep{Kleesiek2023AnOnly}.

Sin embargo, mientras que ChatGPT ha sorprendido al mundo con su habilidad conversacional y AI, ha surgido una distinción crítica entre la capacidad de conversación general de ChatGPT y las aplicaciones médicas específicas. La programación y el entrenamiento de ChatGPT están diseñados para conversaciones generales y no para soporte diagnóstico o recomendaciones de tratamiento \citep{Kleesiek2023AnOnly}. Esta delimitación es esencial, especialmente en el ámbito de la salud laboral, donde la precisión y la fiabilidad de la información son vitales.

La discusión sobre la función de ChatGPT en la atención médica, particularmente en el contexto laboral, se centra en dos aspectos críticos: el uso previsto versus el uso real y las expectativas de los desarrolladores en contraposición a las de los usuarios finales \citep{Kleesiek2023AnOnly}. A pesar de que ChatGPT siempre aclara que no es un profesional de la salud, las posibles implicaciones de su uso en la atención médica laboral plantean preguntas sobre la responsabilidad y la precisión del contenido que genera.

Las tecnologías disruptivas como ChatGPT ofrecen tanto amenazas como oportunidades. En el mejor de los casos, pueden surgir sinergias entre humanos y computadoras, como la combinación de capacidades humanas y computacionales para lograr objetivos más amplios. Sin embargo, el peligro radica en confiar ciegamente en la tecnología, lo que puede llevar a la desinformación, especialmente en un área tan crítica como la atención médica laboral \citep{Kleesiek2023AnOnly}.

Es innegable que LLMs como ChatGPT tienen un vasto potencial en la atención médica. Desde la generación de texto para completar informes clínicos hasta la interpretación y explicación de otros algoritmos de AI, las aplicaciones son vastas \citep{Kleesiek2023AnOnly}. Sin embargo, es esencial que estos desarrollos se realicen con precaución, especialmente en el ámbito laboral, donde las decisiones basadas en la información proporcionada pueden tener consecuencias significativas para la salud y el bienestar de los trabajadores.

En conclusión, mientras ChatGPT y tecnologías similares ofrecen posibilidades emocionantes en el ámbito de la salud laboral, es crucial que se utilicen con discernimiento, y se comprendan plenamente sus limitaciones y potencialidades. El futuro puede ser prometedor, pero es esencial que, como sociedad, guiemos su desarrollo de manera responsable \citep{Kleesiek2023AnOnly}.



\subsubsection{La Integración de Inteligencia Artificial en la Salud Ocupacional: Una Evaluación Crítica de ChatGPT en Escenarios Clínicos y de Investigación}
El avance tecnológico ha llevado a la incorporación de la inteligencia artificial (IA) en múltiples dominios de la atención médica. Un análisis reciente realizado \citet{Cascella2023EvaluatingScenarios} exploró la viabilidad de una de estas herramientas de IA, específicamente ChatGPT, en diferentes escenarios clínicos y de investigación. Esta exploración es especialmente relevante en el contexto de la salud y bienestar de los trabajadores, ya que la seguridad y salud laboral no solo comprenden la prevención de accidentes, sino también la salud integral del trabajador, lo que abarca aspectos médicos y de atención sanitaria.

\citet{Cascella2023EvaluatingScenarios} examinaron la capacidad de ChatGPT para comprender y razonar sobre temas de salud pública, especialmente en relación con el concepto de seniority y cómo se mide objetivamente la edad biológica de una persona. El estudio reveló que el chatbot podía proporcionar definiciones precisas, categorizar a los adultos mayores en subgrupos según su edad, y mencionar métodos comunes para estudiar la senioridad desde una perspectiva biológica, como el desarrollo dental y esquelético, la longitud de los telómeros y la función cognitiva, entre otros \citep{Cascella2023EvaluatingScenarios}.

Más aún, ChatGPT demostró su capacidad para citar estudios clínicos relevantes que respaldaban sus respuestas, lo que sugiere su potencial en la exploración de la literatura y la generación de nuevas hipótesis de investigación. Sin embargo, \citet{Cascella2023EvaluatingScenarios} también identificaron posibles mal uso de la herramienta, como la generación de noticias falsas o información errónea, y el potencial de ChatGPT para reproducir sesgos presentes en los datos con los que fue entrenado.

Dada la importancia de la salud ocupacional y la necesidad de sistemas y productos que mejoren la salud de los trabajadores, las aplicaciones de IA como ChatGPT emergen como herramientas cruciales. Aun así, es esencial que la comunidad científica comprenda sus límites y capacidades para garantizar su uso efectivo y seguro en contextos clínicos y de investigación \citep{Cascella2023EvaluatingScenarios}.

En conclusión, la incorporación de herramientas de IA en la salud y bienestar de los trabajadores presenta un horizonte prometedor, pero también plantea desafíos significativos. Los hallazgos de \citet{Cascella2023EvaluatingScenarios} proporcionan una visión crítica de una de estas herramientas, ChatGPT, resaltando su potencial y limitaciones en escenarios de atención médica.






% Esta sección da cuenta del estado en el que se encuentra la investigación sobre el tema que se está explorando con el proyecto de grado. Tiene como objetivo revisar y analizar el conocimiento acumulado alrededor del problema, y evidenciar cuál es el estado actual de la solución a un problema respecto al problema que se desea abordar. 

% Esta sección presenta trabajos previos (estudios o implementaciones) que abordan el problema de forma similar, da confianza sobre el conocimiento del autor de referentes anteriores así como permite que no se repitan estudios sobre asuntos explorados previamente.

% \textbf{Nota:} \textit{En el anteproyecto este análisis puede ser más superficial pero a medida que lo haga mejor podrá reutilizar más para su documento final.}

% \subsubsection*{¿Qué incluir?}
% Piense en los siguientes temas:
% \begin{itemize}
%     \item ¿Cómo se ha resuelto el problema?
%     \item ¿Quienes han resuelto el problema?
%     \item ¿Qué aspectos técnicos económicos, culturales normativas estándares se han tenido en cuenta?
%     \item ¿El problema ha sido resuelto en otro contexto?
% \end{itemize}

% \subsubsection*{¿Qué NO debe incluir?}
% \begin{itemize}
%     \item NO incluya ideas propias o reflexiones respecto a cómo solucionar el problema. Facts only. 
%     \item NO describa información de otros trabajos o problemáticas que usted NO va a abordar 
% \end{itemize}

% \subsubsection*{¿Cómo organizarlo?}
% \begin{itemize}
%     \item Definición de un conjunto de criterios que van a usarse para comparar los trabajos de otros
%     \item Una descripción corta de cada propuesta - previas de solución del problema. Resalte en cada una ventajas y desventajas. 
%     \item  Indique de forma clara: ¿Por qué las propuestas y soluciones revisadas no sirven en el contexto del estudio y porque no resuelven la pregunta planteada en su proyecto de investigación?
%     \item \textbf{DESEABLE}: haga tablas o gráficas que presenten cuál es el vacío que tiene la situación actual. 
% \end{itemize}


