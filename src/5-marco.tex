\section{Marco teórico de referencia y antecedentes}
% Es una recopilación de información de todo aquello que se haya hecho alrededor del tema propuesto. Sirve como una orientación acerca del enfoque que debe darse al proyecto, porque al acudir a los antecedentes, el proponente puede darse cuenta de cómo ha sido tratado el problema: qué tipos de estudio se han efectuado sobre él, qué modelos y diseños se han utilizado, dónde y cómo se han recolectado los datos, etc. 

% \textbf{La construcción del estado del arte sirve como punto de partida para la realización del proyecto.}

% Para el anteproyecto se recomienda una extensión de máximo 10 páginas para esta sección.

\subsection{Bases Teóricas}
% En esta sección se describen los fundamentos teóricos que sustentan el trabajo de investigación o proyecto de grado, con sus respectivas citas bibliográficas (Es muy importante el manejo riguroso de las citas). 

% Tenga encuenta la siguiente lista de chequeo:
% \begin{itemize}
%     \item Los temas tratados en el marco teórico deben ser relevantes al problema que se está abordando. Estas descripciones no deben ser demasiado extensas ni repetir la teoría que está en los libros, debe presentar los conceptos fundamentales y hacer referencias a libros o artículos donde esos temas se tratan con mayor detalle. 
%     \item Los temas aquí abordados deben ser relevantes con el fin de hacer el documento autocontenido.
%     \item NO asuma que el lector es un experto 
%     \item NO se trata de una enumeración de
% fuentes y conceptos aislados, sino de que presenta de manera artículada los conceptos relevantes para entender la investigación. 
% \item Use las definiciones de los autores "seminals" o los autores referentes del área. 
% \end{itemize}

% Tips de lo que no va:
% \begin{itemize}
% \item  NO incluir material que el lector no necesita para entender lo que sigue
% \item  No incluya material que no se conecta con alguna sección de la tesis
% \item NO incluya temas que rompan el flujo del argumento. Algunas cosas pueden parecer importantes pero podrían ser Anexos. 
% \end{itemize}

\subsection{Estado del Arte}
% Esta sección da cuenta del estado en el que se encuentra la investigación sobre el tema que se está explorando con el proyecto de grado. Tiene como objetivo revisar y analizar el conocimiento acumulado alrededor del problema, y evidenciar cuál es el estado actual de la solución a un problema respecto al problema que se desea abordar. 

% Esta sección presenta trabajos previos (estudios o implementaciones) que abordan el problema de forma similar, da confianza sobre el conocimiento del autor de referentes anteriores así como permite que no se repitan estudios sobre asuntos explorados previamente.

% \textbf{Nota:} \textit{En el anteproyecto este análisis puede ser más superficial pero a medida que lo haga mejor podrá reutilizar más para su documento final.}

% \subsubsection*{¿Qué incluir?}
% Piense en los siguientes temas:
% \begin{itemize}
%     \item ¿Cómo se ha resuelto el problema?
%     \item ¿Quienes han resuelto el problema?
%     \item ¿Qué aspectos técnicos económicos, culturales normativas estándares se han tenido en cuenta?
%     \item ¿El problema ha sido resuelto en otro contexto?
% \end{itemize}

% \subsubsection*{¿Qué NO debe incluir?}
% \begin{itemize}
%     \item NO incluya ideas propias o reflexiones respecto a cómo solucionar el problema. Facts only. 
%     \item NO describa información de otros trabajos o problemáticas que usted NO va a abordar 
% \end{itemize}

% \subsubsection*{¿Cómo organizarlo?}
% \begin{itemize}
%     \item Definición de un conjunto de criterios que van a usarse para comparar los trabajos de otros
%     \item Una descripción corta de cada propuesta - previas de solución del problema. Resalte en cada una ventajas y desventajas. 
%     \item  Indique de forma clara: ¿Por qué las propuestas y soluciones revisadas no sirven en el contexto del estudio y porque no resuelven la pregunta planteada en su proyecto de investigación?
%     \item \textbf{DESEABLE}: haga tablas o gráficas que presenten cuál es el vacío que tiene la situación actual. 
% \end{itemize}


