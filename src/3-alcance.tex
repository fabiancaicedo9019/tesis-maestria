\section{Alcance}
\label{sec:alcance}
El presente proyecto se centra en la integración de componentes de Inteligencia Artificial (IA) en soluciones de software específicas para el sector de Seguridad y Salud en el Trabajo (SST). Se busca comprender las funcionalidades y características de los sistemas de software actuales en este sector, identificando áreas de oportunidad donde la IA pueda potenciar o mejorar dichas funcionalidades.

Es importante dentro del alcance de este trabajo la creación de un compendio de buenas prácticas para la incorporación de funcionalidades de IA, esperando que sirva de guía para futuras implementaciones. De forma complementaria, se propondrá una lista de componentes arquitectónicos que podrían considerarse en la integración de la IA dentro de las soluciones de software ya existentes.

Se definirán criterios específicos para asegurar que la IA entregue respuestas precisas y relevantes, con especial atención en la elaboración y gestión del contexto que requieren los prompts enfocados en SST. Además, se identificarán y propondrán las adaptaciones arquitectónicas necesarias en las arquitecturas de software preexistentes para facilitar la integración de la IA, lo cual es esencial para una incorporación exitosa y eficiente.

Para validar la propuesta del marco de trabajo, se desarrollará un prototipo funcional implementado en una plataforma web modular. Sin embargo, es importante señalar lo que este proyecto no abarcará:
\begin{itemize}
    \item No se desarrollarán nuevos componentes de IA desde cero; en su lugar, se utilizarán modelos ya existentes para lograr la interoperabilidad con las soluciones de software actuales.
    \item Solo se examinarán tres estilos arquitectónicos: arquitecturas monolíticas, microservicios y basadas en servicios, excluyendo cualquier otro estilo arquitectónico.
    \item Las buenas prácticas que se revisarán estarán únicamente enfocadas en la gestión de recursos computacionales, la selección de herramientas o librerías y modelos de IA pertinentes, sin extenderse a otras áreas de práctica que puedan influir en la integración de la IA.
    \item Este trabajo se centrará predominantemente en la fase de diseño del desarrollo de software, particularmente en lo que respecta a la arquitectura de software, sin abarcar las fases posteriores de implementación y mantenimiento.
\end{itemize}

La adopción de este enfoque selectivo asegura que el proyecto se desarrolle dentro de un marco delimitado, lo que facilitará la consecución de los objetivos propuestos, asegurándose de que sean prácticos y realistas. Estableciendo estas limitaciones de manera explícita, se configura un contorno para el trabajo de investigación, lo cual contribuirá significativamente al ámbito de SST. Así, el documento resultante no solo reflejará una investigación dirigida y metódica, sino que también se espera que sea una contribución valiosa, ofreciendo perspectivas y soluciones aplicables al sector específico de SST.







% El alcance de este proyecto se centra en la integración de componentes de Inteligencia Artificial (IA) en soluciones de software específicas para el sector de seguridad y salud en el trabajo. Se busca comprender a fondo las funcionalidades y características de los sistemas de software actuales en este sector, identificando áreas de oportunidad donde la IA pueda potenciar o mejorar dichas funcionalidades.

% La elaboración de una lista de buenas prácticas para la incorporación de funcionalidades de IA es un aspecto importante de este proyecto, y se espera que sirva como guía para futuras implementaciones. Paralelamente, se propondrá una lista de componentes arquitectónicos potenciales que podrían ser considerados en la integración de IA dentro de soluciones de software existentes.

% El trabajo también implica la definición de criterios específicos para garantizar que la IA proporcione respuestas precisas y relevantes, con un enfoque particular en la elaboración y gestión de prompts. Esto es fundamental para mejorar la efectividad de las soluciones propuestas y evitar malentendidos o inexactitudes en las respuestas de la IA.

% En cuanto a las adaptaciones arquitectónicas, este proyecto identificará y propondrá cambios necesarios en las arquitecturas de software preexistentes para facilitar la integración de componentes de IA. Este aspecto es vital para asegurar una integración óptima y eficiente, aprovechando al máximo las capacidades de la IA.

% El desarrollo de un prototipo funcional implementado en una plataforma web modular servirá como caso de estudio para validar la propuesta del marco de trabajo. Es importante destacar que el objetivo no es desarrollar nuevos componentes de IA, sino demostrar cómo se pueden incorporar eficazmente componentes existentes en soluciones de software.

% Este proyecto no tiene como objetivo desarrollar nuevos componentes de IA, ni tampoco hará públicos en detalle los componentes desarrollados o implementados en el prototipo que pertenezcan o sean utilizados por la empresa, protegiendo así la propiedad intelectual y los intereses comerciales.

% En resumen, el alcance de este trabajo abarca desde la comprensión y mejora de las funcionalidades de los sistemas de software actuales en el sector de seguridad y salud en el trabajo, hasta la propuesta de adaptaciones arquitectónicas y el desarrollo de un prototipo funcional para demostrar la integración efectiva de la IA en estas soluciones.