\section{Alcance}
La incorporación de componentes de Inteligencia Artificial (IA) en soluciones orientadas al sector de seguridad y salud en el trabajo representa un desafío que va más allá de la mera implementación técnica. Se entiende la necesidad de profundizar en el entendimiento de las características y funcionalidades que los actuales sistemas de software en este sector ofrecen. Este trabajo se propone identificar las áreas en las que la IA podría potenciar o mejorar dichas funcionalidades, estableciendo así un punto de partida firme para cualquier intento de incorporación.

El tema de la ética y la privacidad se ha convertido en una preocupación en la aplicación de la IA, sobre todo cuando se trata de campos tan delicados como la seguridad y salud en el trabajo. No es solo una cuestión de garantizar que los sistemas sean justos y transparentes, sino también de anticipar posibles riesgos y diseñar estrategias para mitigarlos. Por ello, este trabajo se sumergirá en las intrincadas aguas de las implicaciones éticas, identificando y proponiendo medidas que aseguren un despliegue de IA responsable y centrado en el humano.

En paralelo, la gestión de los prompts en la IA, especialmente en un sector tan específico, es fundamental para garantizar la efectividad de cualquier solución propuesta. Definir los aspectos clave de esta gestión permitirá que la implementación de IA sea precisa y relevante, evitando malentendidos o inexactitudes que podrían surgir de una gestión deficiente de prompts.

Sin embargo, la verdadera culminación de esta investigación se encuentra en la identificación de las adaptaciones arquitectónicas requeridas para una integración óptima de la IA en sistemas existentes. Al considerar la arquitectura y la funcionalidad ya presentes en las soluciones de software del sector, se pueden identificar los desafíos y oportunidades únicas que este proyecto abordará.

Dado que la investigación culminará con el desarrollo de un prototipo funcional, es crucial mencionar que el propósito no es crear nuevos componentes de IA desde cero. En su lugar, se utilizarán componentes existentes, y el foco estará en demostrar cómo incorporar eficazmente estos componentes dentro de una aplicación web modular. Este prototipo actuará como una demostración tangible de los principios y estrategias discutidos a lo largo de la investigación.

Finalmente, es esencial subrayar las limitaciones de este proyecto. Aunque se investigará profundamente la integración de IA, no se desarrollarán nuevos componentes de IA. Además, cualquier componente desarrollado o implementado en el prototipo que pertenezca o sea usado por la empresa no se hará público en detalle, resguardando así la propiedad intelectual y los intereses comerciales.

En suma, este trabajo representa una profunda inmersión en la convergencia entre la IA y el sector de seguridad y salud en el trabajo, buscando no solo entender las oportunidades y desafíos actuales, sino también establecer un camino práctico y ético hacia la futura integración de la IA en este ámbito.
