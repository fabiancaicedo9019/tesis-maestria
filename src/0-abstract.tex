%%%%%%%%%%%%%%%%
% ABSTRACT
%%%%%%%%%%%%%%%%

\section*{Resumen}
A pesar de los evidentes avances en el área de la Inteligencia Artificial (IA), su integración efectiva en soluciones de software orientadas a la Seguridad y Salud en el Trabajo (SST) presenta desafíos que abarcan desde aspectos técnicos hasta cuestiones éticas y de privacidad, y demandan una comprensión profunda y enfoques adaptados para asegurar implementaciones exitosas que realmente beneficien a los usuarios finales y a las organizaciones involucradas.

De ahí que, la presente propuesta de investigación propone un marco de trabajo para incorporar componentes de IA en arquitecturas de software preexistentes con énfasis en SST. El marco de trabajo se compone de prácticas recomendadas, componentes arquitectónicos y criterios para una integración eficaz de una IA, buscando no solo la adaptación técnica sino también el aprovechamiento máximo de la IA para garantizar su impacto y perdurabilidad. 

En particular, como caso de estudio, se seleccionará un proyecto de desarrollo de software el cual incluya en sus requerimientos funcionales la necesidad de incorporar componentes de IA. La Corporacion Talentum es una entidad prominente en la implementaci on de proyectos gubernamentales en Colombia.

%Los resultados esperados son un manual de buenas prácticas, diagramas UML, y el modelo C4, junto con un prototipo que evidencie la funcionalidad del marco en situaciones reales. Este trabajo contribuye al campo de IA y arquitectura de software, proveyendo guías y herramientas para la integración de IA, resaltando la importancia de una implementación orientada y consciente en ámbitos críticos como la seguridad y salud laboral.


% La propuesta de investigaci\'on presentada propone un marco de trabajo integral y detallado para la incorporación efectiva de componentes basados en Inteligencia Artificial (IA) en arquitecturas de software ya existentes, poniendo especial atención en el ámbito de la seguridad y salud laboral. Mediante un análisis y revisión de la literatura existente, se examina la problemática asociada a la integración de la IA en sistemas preexistentes, identificando así los retos técnicos, arquitectónicos y contextuales que implica la implementación de esta tecnología.

% En este marco, se utiliza como caso de estudio la Corporación Talentum, una entidad prominente en la implementación de proyectos gubernamentales en Colombia. Esta corporación afronta desafíos significativos para integrar productos de IA en sus soluciones, resaltando la urgente necesidad de establecer un conjunto estandarizado de buenas prácticas y directrices.

% La propuesta de investigaci\'on propone un marco de trabajo que comprende una serie de buenas prácticas, componentes arquitectónicos potenciales y criterios específicos destinados a lograr una integración efectiva de la IA. El objetivo de este marco no es únicamente facilitar la adaptación técnica, sino también maximizar los beneficios derivados de la IA, asegurando así su relevancia y sostenibilidad a largo plazo. Como parte de la propuesta, se presenta un prototipo funcional web y modular, que sirve como caso práctico para validar la aplicabilidad del marco de trabajo propuesto.

% Se anticipa que los resultados de este trabajo incluyan un documento que defina el marco de trabajo y proporcione buenas prácticas para la integración de la IA, acompañado de un conjunto de diagramas UML y del modelo C4 para ilustrar los componentes arquitectónicos sugeridos. Se espera, además, que el prototipo funcional demuestre de manera tangible la utilidad y aplicabilidad del marco de trabajo en contextos reales.

% Este trabajo aporta significativamente al campo de la IA y la arquitectura de software, ofreciendo una guía práctica y un conjunto de herramientas útiles para la integración de productos basados en IA en soluciones ya existentes. De igual manera, subraya la necesidad de una implementación orientada de la IA, especialmente en sectores de crítica importancia como lo es la seguridad y salud en el trabajo.




\paragraph*{}{\textbf{Palabras Clave:}}
Inteligencia Artificial (IA), Seguridad y Salud en el Trabajo (SST), Ingenier\'ia de Software (IS), Marco de Trabajo, Prompt.

\section*{Abstract}
Despite the evident advances in the area of Artificial Intelligence (AI), its effective integration into software solutions focused on Occupational Health and Safety (OSH) presents challenges that range from technical aspects to ethical and privacy issues and demand deep understanding and tailored approaches to ensure successful implementations that truly benefit end users and the organizations involved.

Hence, the present research proposal proposes a framework to incorporate AI components into pre-existing software architectures with an emphasis on SST. The framework is made up of recommended practices, architectural components and criteria for an effective integration of an AI, seeking not only technical adaptation but also the maximum use of AI to guarantee its impact and durability.

In particular, as a case study, a software development project will be selected which includes in its functional requirements, the need to incorporate AI components. The Talentum Corporation is a
Prominent entity in the implementation of government projects in
Colombia.


%This research proposal comprises a framework for incorporating Artificial Intelligence (AI) into pre-existing software architectures, with an emphasis on occupational health and safety. It analyzes the challenges of integrating AI into established software systems, covering technical, structural and contextual issues. The central case study is the Talentum Corporation in Colombia, highlighting the need for standardized practices for AI insertion in government projects.

%The proposed framework consists of best practices, architectural components and criteria for effective AI integration, seeking not only technical adaptation but also the maximum use of AI to ensure its impact and durability. It includes a modular web prototype to validate its feasibility.

%The expected results are a best practices manual, UML diagrams, and the C4 model, along with a prototype that evidences the functionality of the framework in real situations. This work contributes to the field of AI and software architecture, providing guidelines and tools for AI integration, highlighting the importance of a targeted and conscious implementation in critical areas such as occupational health and safety.

\paragraph*{}{\textbf{Keywords:}}
Artificial Intelligence (AI), Occupational Safety and Health (OSH), Software Engineering (SE), Framework, Prompt.