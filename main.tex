
%% Formato para la presentaci\'on de  proyectos de grado. 
%% Maestría en Ingeniería de Software
%% Pontificia Universidad Javeriana
%% Elaborado por Angela Villota y Luisa Rincón
%% Inspirado en Formato elaborado para carrera de Ing de Sistemas
%% V 1.0 Abril - 2022
\documentclass[12pt]{article}
\usepackage[utf8]{inputenc}
\usepackage{geometry}
\geometry{letterpaper}
\usepackage[spanish,es-tabla]{babel}
% \usepackage{cite}
\usepackage{titling}
\usepackage{setspace}
\usepackage{graphicx}
\graphicspath{{img/}} %ruta de la carpeta en donde estaràn las imágenes
\usepackage{blindtext}
\usepackage[hidelinks]{hyperref}
\setlength{\marginparwidth}{2cm}
\usepackage{natbib} %% para citaciones
\usepackage{todonotes}
\usepackage[format=hang,font=small,labelfont=bf]{caption}

\usepackage{pgfgantt}
\usepackage{lscape}
\usepackage{rotating}

\usepackage{enumitem}

% *** ALIGNMENT PACKAGES ***
%
\usepackage{array}
\usepackage{booktabs}
\usepackage{pdflscape}
\usepackage{multirow}
\usepackage{float}
\usepackage{longtable}
\usepackage{dblfloatfix}
\usepackage{subfig}

\onehalfspacing
\setcounter{secnumdepth}{5}
\setcounter{tocdepth}{2}

\usepackage{xcolor}
\definecolor{light-gray}{gray}{0.95}
\newcommand{\code}[1]{\colorbox{light-gray}{\texttt{#1}}}

\renewcommand\maketitlehooka{\null\mbox{}\vfill}
\renewcommand\maketitlehookd{\vfill\null}
\DeclareUnicodeCharacter{0301}{\hspace{-1ex}\'{ }}

\begin{document}
%crear titulo
%\maketitle

%%%%%%%%%%%%%%%%%
% Portada del Anteproyecto
%%%%%%%%%%%%%%%%%
\begin{center}
\thispagestyle{empty}
\vspace*{-2.2cm}
\begin{center}
    \includegraphics[width=9cm]{pujlogo}~\\[1cm]
\end{center}
\textbf{\huge
Marco de trabajo para incorporación de buenas prácticas y directrices para la integración de nuevos productos con Inteligencia Artificial en las arquitecturas existentes en la Corporación Talentum}\\[1.2cm]
% \todo[inline]{Marco de trabajo para incorporación de buenas prácticas y directrices para la integración de nuevos productos con Inteligencia Artificial en las arquitecturas existentes en la Corporación Talentum}
% T\'{\i}tulo de la tesis  o trabajo de investigación}\\[1.75cm]
\Large\textbf{Fabian Andres Caicedo Cuellar}\\[1cm]
\small Anteproyecto presentada(o) como requisito parcial para optar al
t\'{\i}tulo de:\\
\textbf{Magister en Ingenier\'{\i}a de Software}\\[1cm]
Director(a):\\
T\'{\i}tulo (Ph.D., MSc) y nombre del director(a)\\[1cm]

Pontificia Universidad Javeriana Cali\\
Facultad de Ingeniería\\
Departamento de Electrónica y Ciencias de la Computación\\
Cali, Colombia\\
\today\\
\end{center}

\newpage
%%%%%%%%%%%%%%%%%
% Ficha resumen
%%%%%%%%%%%%%%%%%
\thispagestyle{empty}
\begin{center}
    \Large{Ficha Resumen \\ Anteproyecto de Trabajo de Grado}
\end{center}

\textbf{Marco de trabajo para la integración de herramientas de IA en los productos de software para el área de Seguridad y Salud en el Trabajo desarrollados por la Corporación Talentum}
\begin{enumerate}
    \item Área de trabajo: Ingeniería y tecnología
    \item Tipo de proyecto: Aplicado
    \item Estudiante: Fabián Andrés Caicedo Cuellar
    \item Correo electrónico: fabiancaicedo@javerianacali.edu.co
    \item Dirección y teléfono: Carrera 23C \#10-02 / 3176367317
    \item Director: Oscar Orlando Ceballos Argote, Ph.D.
    \item Vinculación del director: Hora C\'atedra
    \item Correo electrónico del director: oscar.ceballos@javerianacali.edu.co
    %\item Co-Director (Si aplica):
    \item Grupo o empresa que lo avala (Si aplica): Corporación Talentum
    % \item Otros grupos o empresas:
    \item Palabras clave(al menos 5): Inteligencia Artificial (IA), Seguridad y Salud en el Trabajo
(SST), Ingeniería de Software (IS), Marco de Trabajo, Prompt.
    \item Fecha de inicio: 1 de Enero de 2024
    \item Duración estimada (en meses): 6 meses
    \item Resumen:  La presente propuesta de investigación propone un marco de trabajo para la incorporación efectiva de componentes basados en Inteligencia Artificial (IA) en arquitecturas de software ya existentes, poniendo especial atención en el ámbito de la Seguridad y Salud en el Trabajo (SST). Mediante un análisis y revisión de la literatura existente, se examina la problemática asociada a la integración de la IA en sistemas preexistentes, identificando así los retos técnicos, arquitectónicos y contextuales que implica la implementación de esta tecnología.

En particular, se utiliza como caso de estudio la Corporación Talentum, una entidad prominente en la implementación de proyectos gubernamentales en Colombia. Esta corporación afronta desafíos significativos para integrar productos de IA en sus soluciones, resaltando la urgente necesidad de establecer un conjunto estandarizado de buenas prácticas y directrices.

%La propuesta de investigación propone un marco de trabajo que comprende una serie de buenas prácticas, componentes arquitectónicos potenciales y criterios específicos destinados a lograr una integración efectiva de la IA. El objetivo de este marco no es únicamente facilitar la adaptación técnica, sino también maximizar los beneficios derivados de la IA, asegurando así su relevancia y sostenibilidad a largo plazo. Como parte de la propuesta, se presenta un prototipo funcional web y modular, que sirve como caso práctico para validar la aplicabilidad del marco de trabajo propuesto.

Como resultados esperados se construirá un documento que defina el marco de trabajo a través de buenas prácticas para la integración de la IA, acompañado de un conjunto de diagramas UML y del modelo C4 para ilustrar los componentes arquitectónicos sugeridos. Se espera, además, desarrollar un prototipo funcional que demuestre de manera tangible la utilidad y aplicabilidad del marco de trabajo en contextos reales.

%Este trabajo aporta significativamente al campo de la IA y la arquitectura de software, ofreciendo una guía práctica y un conjunto de herramientas útiles para la integración de productos basados en IA en soluciones ya existentes. De igual manera, subraya la necesidad de una implementación orientada de la IA, especialmente en sectores de crítica importancia como lo es la seguridad y salud en el trabajo.

\end{enumerate}

\newpage

%%%%%%%%%%%%%%%%%
% Indices y tablas
%%%%%%%%%%%%%%%%%

\tableofcontents
\listoffigures
\listoftables
% \listoftodos
\newpage

%%%%%%%%%%%%%%%%%
% Resumen / abstract
%%%%%%%%%%%%%%%%%
%%%%%%%%%%%%%%%%
% ABSTRACT
%%%%%%%%%%%%%%%%

\section*{Resumen}

\paragraph*{}{\textbf{Palabras Clave:}}
Inteligencia Artificial (IA), Seguridad y Salud en el Trabajo (SST), Ingenier\'ia de Software (IS).

\section*{Abstract}

\paragraph*{}{\textbf{Keywords:}}
Artificial Intelligence (AI), Occupational Safety and Health (OSH), Software Engineering (SE). 
\newpage

%%%%%%%%%%%%%%%%%
% Introduccion
%%%%%%%%%%%%%%%%%
\section{Introducción}

En el ámbito global, la Inteligencia Artificial (IA) ha revolucionado diversos sectores, facilitando la toma de decisiones, optimizando procesos y generando soluciones innovadoras para problemas complejos. Particularmente, la IA ha demostrado su potencial en la mejora de sistemas de software diseñados para sectores específicos como lo es el sector de la Seguridad y Salud en el Trabajo (SST), especialmente en contextos como el colombiano que se encuentra en la cúspide de esta revolución tecnológica donde organizaciones como la Corporación Talentum aspiran a liderar iniciativas que generen un impacto positivo en el bienestar laboral.

No obstante, emerge una problemática significativa: a pesar de los avances en IA, su integración efectiva en soluciones de software orientadas a la SST presenta desafíos que abarcan desde aspectos técnicos hasta cuestiones éticas y de privacidad, y demandan una comprensión profunda y enfoques adaptados para asegurar implementaciones exitosas que realmente beneficien a los usuarios finales y a las organizaciones involucradas.

Es imperativo abordar esta problemática, dado que las soluciones adecuadas poseen el potencial de transformar cómo las organizaciones gestionan la seguridad y salud en el trabajo. Una implementación efectiva de IA puede facilitar la identificación temprana de riesgos, optimizar respuestas y promover ambientes de trabajo más seguros y saludables. Por lo tanto, resulta esencial establecer estrategias y métodos claros que permitan maximizar los beneficios de la IA en este sector.

En este contexto, se presenta esta propuesta con el propósito de establecer dichas estrategias y métodos para una efectiva incorporación de componentes de IA en soluciones dirigidas al sector mencionado. Con este fin, no solo se plantea el objetivo de proporcionar directrices claras y evidencia palpable de cómo la IA puede enriquecer este campo, sino que también se espera producir resultados concretos: un conjunto de principios arquitectónicos fundamentales para considerar al diseñar software con requisitos relacionados con la IA; un prototipo funcional web y modular que demuestre la aplicación del marco de trabajo documentado; y finalmente, un documento detallado que defina el marco de trabajo y ofrezca buenas prácticas, convenciones y requisitos esenciales para la adecuada integración de aplicaciones con componentes de IA.

Para lograr una comprensión profunda y enfrentar la problemática destacada, se realizará una revisión de literatura, fundamentado en una revisión de documentos existentes, el análisis de datos pertinentes y la aplicación de metodologías apropiadas. De esta manera, se proporcionará no solo una base sólida para futuras implementaciones de IA en el ámbito de SST, sino también en productos que incorporen IA en sus requerimientos funcionales.

La estructura de este documento se ha diseñado para facilitar la comprensión y el análisis de cada aspecto involucrado en la incorporación de la Inteligencia Artificial al sector de la Seguridad y Salud en el Trabajo (SST). Comenzando con una definición del problema en la Sección 2, se elabora sobre la situación actual y las preguntas específicas que guían este estudio. Los Objetivos del proyecto, tanto generales como específicos, así como los resultados esperados, se exponen en la Sección 3, mientras que el Alcance del estudio se detalla en la Sección 4. La justificación del trabajo de grado se presenta en la Sección 5, seguida por el Marco teórico de referencia y antecedentes, que incluye las Bases Teóricas y el Estado del Arte en la Sección 6, proporcionando un contexto esencial para la investigación. La metodología adoptada para llevar a cabo esta investigación se describe en la Sección 7. Los recursos necesarios para la ejecución del proyecto, incluyendo los Recursos Humanos, Bibliográficos y Tecnológicos se enumeran en la Sección 8. Finalmente, la Sección 9 esboza el Cronograma de actividades que guiará el desarrollo de la investigación, seguido por las Referencias Bibliográficas y el Glosario de Términos en las secciones 10 y 11, respectivamente.

Para sintetizar y visualizar la estructura integral de la investigación y sus componentes clave, se presenta la siguiente figura. Este esquema gráfico tiene como objetivo proporcionar una representación clara y concisa de la propuesta de proyecto, destacando cómo cada elemento interactúa y contribuye al desarrollo de soluciones innovadoras y eficaces para la integración de la IA en el sector de la SST.

\begin{figure}[H]
\centering
\rotatebox{90}{\includegraphics[width=1.15\textwidth]{img/resumen_anteproyecto.pdf}}
\caption{Diagrama resumen de la propuesta de investigación.}
\label{fig:mi_figura}
\end{figure} 
\newpage

%% Problema
\section{Definición del problema}

\subsection{Planteamiento del problema}
En la era digital actual, la Inteligencia Artificial (IA) ha transformado rápidamente múltiples sectores, se puede decir que desde la atención médica y la manufactura hasta el entretenimiento y la logística. Su virtud para analizar grandes volúmenes de datos, hacer predicciones precisas y automatizar tareas complejas ha demostrado ser un activo invaluable. 

En el \'area del Desarrollo de Software, la integración de la IA presenta oportunidades únicas para mejorar la eficiencia, la funcionalidad y la experiencia del usuario entre otros aspectos. De ah\'i que, muchas organizaciones están ansiosas por adoptar y beneficiarse de la integraci\'on de las capacidades de la IA en sus productos. La integración eficiente y efectiva de productos basados en IA en sistemas existentes es esencial para mantener la relevancia en un mercado en constante cambio y para aprovechar al máximo los beneficios o ventajas que ofrece la IA. Esta integración no solo implica la adaptación técnica, sino también la consideración de cómo los productos de IA pueden añadir valor real a los usuarios y a la organización \citep{Cui2022ConstructionIntelligence}. 

Sin embargo, estas mismas organizaciones tambi\'en deben enfrentarse a diferentes desafíos como la falta de un conjunto estandarizado de buenas prácticas que gu\'ien la integraci\'on de soluciones de IA en arquitecturas de software preexistentes \citep{Wang2016ImplementingOutlook}. Esta ausencia puede dar lugar a incompatibilidades, redundancias, ineficiencias y, en el peor de los casos, a fallas en el sistema. Sin una orientación clara y prácticas estandarizadas, las organizaciones pueden encontrarse con sistemas sobre complicados, costosos y que no cumplen con las expectativas. 

Aunque la literatura ha abordado ampliamente las capacidades y aplicaciones de la IA, existe una notoria falta de investigación exhaustiva sobre la estrategia de integración de estas soluciones en el mercado tecnológico \citep{Cui2022ConstructionIntelligence}. Esta laguna en la investigación puede dificultar a las organizaciones la toma de decisiones informadas sobre cómo abordar la integración de IA.

%Adicionalmente, con la rápida evolución del mercado y la tecnología, las organizaciones se ven presionadas a adaptarse rápidamente a las necesidades cambiantes \citep{Wang2016ImplementingOutlook}. Sin un marco de referencia sólido, esta adaptación puede volverse reactiva en lugar de proactiva, lo que puede llevar a soluciones temporales o inadecuadas.

En el contexto colombiano, la Corporación Talentum ha consolidado una sólida reputación como ejecutora de proyectos gubernamentales, tanto tecnológicos como no tecnológicos. Estos proyectos, destinados a generar un alto impacto en diferentes sectores gubernamentales, resaltan la importancia de la Seguridad y Salud en el Trabajo. El objetivo primordial de la corporación es mejorar sus procesos de seguridad y salud en el trabajo. No obstante, la Corporación Talentum aspira a ejecutar proyectos gubernamentales con productos propios que incorporen Inteligencia Artificial. La intención es respaldar y ofrecer mejoras en los procesos de estas entidades, creyendo que, al hacerlo, indirectamente contribuirán a mejorar las condiciones y el ambiente laboral. Al promover el bienestar de los empleados a través de estas iniciativas, la corporación manifiesta su compromiso con la salud laboral en Colombia.

A lo largo de su trayectoria, la Corporación Talentum ha reconocido la trascendencia de la Inteligencia Artificial en la configuración de soluciones de vanguardia. Sin embargo, enfrenta desafíos considerables en su intento de incorporar productos de IA, acentuando la sensación de quedar rezagados frente a otras entidades que ya han adoptado estas tecnologías como parte integral de sus ofertas. Esta situación no solo se traduce en una potencial desventaja competitiva, sino que también refleja un desaprovechamiento de oportunidades para aportar valor innovador en el sector de seguridad y salud en el trabajo. 

La falta de un conjunto estandarizado de buenas pr\'acticas para la integración de la Inteligencia Artificial, enmarcadas en un marco de trabajo, ha supuesto un desafío para la Corporación Talentum, lo cual podría influir negativamente en su posicionamiento como ejecutora líder de proyectos gubernamentales en Colombia. Por lo tanto, se hace imperativo investigar y desarrollar un marco de trabajo que proporcione un conjunto de directrices base y mejores prácticas para la integración de productos basados en IA en arquitecturas de software existentes. Esta necesidad no solo es esencial para maximizar los beneficios de la IA, sino también para garantizar la viabilidad, la escalabilidad y la relevancia a largo plazo de las soluciones tecnológicas cada vez más orientado hacia la IA.

\subsection{Formulación del problema}

%En la actualidad, la influencia creciente de la inteligencia artificial (IA) en el desarrollo de software exige que las organizaciones, incluida la Corporación Talentum en Colombia, adapten sus estrategias para mantener su competitividad. A pesar de reconocer el potencial de la IA en el sector de seguridad y salud en el trabajo, la corporación enfrenta desafíos en su incorporación debido a la falta de directrices claras y una estrategia bien definida.
Dado el contexto anterior y teniendo en cuenta las necesidades actuales de la Corporación Talentum para integrar soluciones de IA en el sector de la Salud y Seguridad en el Trabajo, surgen las siguientes interrogantes:
\begin{itemize}
    \item ¿Cómo pueden las soluciones orientadas al sector de seguridad y salud en el trabajo incorporar efectivamente componentes de inteligencia artificial en los requerimientos funcionales del software?
    \item ¿Qué buenas prácticas, estándares y componentes arquitectónicos se deben adoptar para facilitar la integración de productos basados en IA en arquitecturas de software preexistentes?
    \item ¿De qué manera se asegura que la integración de IA en las soluciones de software no solo aporte beneficios técnicos y funcionales, sino que también respete los principios éticos del sector de seguridad y salud en el trabajo?
    \item ¿Qué consideraciones arquitectónicas son necesarias para gestionar adecuadamente los prompts de Inteligencia Artificial, asegurando una comunicación eficiente y precisa con sistemas preexistentes?
\end{itemize}

%Con estas preguntas, se busca abordar la problemática central relacionada con la integración eficiente de la IA en las soluciones de software de la Corporación Talentum en el ámbito de la seguridad y salud en el trabajo. El objetivo es identificar áreas clave de desafío y establecer una estructura que guíe la investigación y desarrollo de directrices y mejores prácticas para este proyecto de maestría.
\newpage

%% Objetivos
\section{Objetivos del proyecto}
\subsection{Objetivo General}
Definir buenas prácticas y estrategias para la incorporación de componentes de Inteligencia Artificial en soluciones software orientadas al sector de Seguridad y Salud en el Trabajo. %atendiendo a los requerimientos funcionales específicos del software.

\subsection{Objetivos específicos}
\begin{enumerate}
    \item Determinar las características y funcionalidades actuales de las soluciones de software en el sector de seguridad y salud en el trabajo que pueden beneficiarse o ser mejoradas con la implementación de Inteligencia Artificial.\todo{ajustar no es entendible}
    \item Analizar las implicaciones éticas y de privacidad en la incorporación de la Inteligencia Artificial en sistemas de seguridad y salud en el trabajo, identificando posibles riesgos y estrategias de mitigación.
    \item Definir los aspectos relevantes para la gestión eficiente y precisa de prompts en la Inteligencia Artificial, orientados al sector de seguridad y salud en el trabajo.
    \item Identificar las adaptaciones arquitectónicas requeridas para una integración efectiva de componentes de Inteligencia Artificial en las estructuras y funcionalidades ya existentes en soluciones de software para el sector de seguridad y salud en el trabajo.
\end{enumerate}

\subsection{Resultados esperados}
\begin{enumerate}
    \item Un conjunto de principios arquitectónicos base para considerar al construir un aplicativo nuevo en el que haya requisitos relacionados con la Inteligencia Artificial.
    \item Desarrollo de un prototipo funcional que aplique el marco de trabajo documentado propuesto. Este aplicativo será web y modular, diseñado para permitir la integración de un componente de Inteligencia Artificial.
    \item Al finalizar el proyecto, se espera presentar un documento que defina el marco de trabajo y especifique buenas prácticas, convenciones y los requisitos mínimos necesarios para una adecuada integración de aplicaciones que utilicen componentes de Inteligencia Artificial.
\end{enumerate}
\newpage

%% Alcance
\section{Alcance}
\label{sec:alcance}
El presente proyecto se centra en la integración de componentes de Inteligencia Artificial (IA) en soluciones de software específicas para el sector de Seguridad y Salud en el Trabajo (SST). Se busca comprender las funcionalidades y características de los sistemas de software actuales en este sector, identificando áreas de oportunidad donde la IA pueda potenciar o mejorar dichas funcionalidades.

Es importante dentro del alcance de este trabajo la creación de un compendio de buenas prácticas para la incorporación de funcionalidades de IA, esperando que sirva de guía para futuras implementaciones. De forma complementaria, se propondrá una lista de componentes arquitectónicos que podrían considerarse en la integración de la IA dentro de las soluciones de software ya existentes.

Se definirán criterios específicos para asegurar que la IA entregue respuestas precisas y relevantes, con especial atención en la elaboración y gestión del contexto que requieren los prompts enfocados en SST. Además, se identificarán y propondrán las adaptaciones arquitectónicas necesarias en las arquitecturas de software preexistentes para facilitar la integración de la IA, lo cual es esencial para una incorporación exitosa y eficiente.

Para validar la propuesta del marco de trabajo, se desarrollará un prototipo funcional implementado en una plataforma web modular. Sin embargo, es importante señalar lo que este proyecto no abarcará:
\begin{itemize}
    \item No se desarrollarán nuevos componentes de IA desde cero; en su lugar, se utilizarán modelos ya existentes para lograr la interoperabilidad con las soluciones de software actuales.
    \item Solo se examinarán tres estilos arquitectónicos: arquitecturas monolíticas, microservicios y basadas en servicios, excluyendo cualquier otro estilo arquitectónico.
    \item Las buenas prácticas que se revisarán estarán únicamente enfocadas en la gestión de recursos computacionales, la selección de herramientas o librerías y modelos de IA pertinentes, sin extenderse a otras áreas de práctica que puedan influir en la integración de la IA.
    \item Este trabajo se centrará predominantemente en la fase de diseño del desarrollo de software, particularmente en lo que respecta a la arquitectura de software, sin abarcar las fases posteriores de implementación y mantenimiento.
\end{itemize}

La adopción de este enfoque selectivo asegura que el proyecto se desarrolle dentro de un marco delimitado, lo que facilitará la consecución de los objetivos propuestos, asegurándose de que sean prácticos y realistas. Estableciendo estas limitaciones de manera explícita, se configura un contorno para el trabajo de investigación, lo cual contribuirá significativamente al ámbito de SST. Así, el documento resultante no solo reflejará una investigación dirigida y metódica, sino que también se espera que sea una contribución valiosa, ofreciendo perspectivas y soluciones aplicables al sector específico de SST.







% El alcance de este proyecto se centra en la integración de componentes de Inteligencia Artificial (IA) en soluciones de software específicas para el sector de seguridad y salud en el trabajo. Se busca comprender a fondo las funcionalidades y características de los sistemas de software actuales en este sector, identificando áreas de oportunidad donde la IA pueda potenciar o mejorar dichas funcionalidades.

% La elaboración de una lista de buenas prácticas para la incorporación de funcionalidades de IA es un aspecto importante de este proyecto, y se espera que sirva como guía para futuras implementaciones. Paralelamente, se propondrá una lista de componentes arquitectónicos potenciales que podrían ser considerados en la integración de IA dentro de soluciones de software existentes.

% El trabajo también implica la definición de criterios específicos para garantizar que la IA proporcione respuestas precisas y relevantes, con un enfoque particular en la elaboración y gestión de prompts. Esto es fundamental para mejorar la efectividad de las soluciones propuestas y evitar malentendidos o inexactitudes en las respuestas de la IA.

% En cuanto a las adaptaciones arquitectónicas, este proyecto identificará y propondrá cambios necesarios en las arquitecturas de software preexistentes para facilitar la integración de componentes de IA. Este aspecto es vital para asegurar una integración óptima y eficiente, aprovechando al máximo las capacidades de la IA.

% El desarrollo de un prototipo funcional implementado en una plataforma web modular servirá como caso de estudio para validar la propuesta del marco de trabajo. Es importante destacar que el objetivo no es desarrollar nuevos componentes de IA, sino demostrar cómo se pueden incorporar eficazmente componentes existentes en soluciones de software.

% Este proyecto no tiene como objetivo desarrollar nuevos componentes de IA, ni tampoco hará públicos en detalle los componentes desarrollados o implementados en el prototipo que pertenezcan o sean utilizados por la empresa, protegiendo así la propiedad intelectual y los intereses comerciales.

% En resumen, el alcance de este trabajo abarca desde la comprensión y mejora de las funcionalidades de los sistemas de software actuales en el sector de seguridad y salud en el trabajo, hasta la propuesta de adaptaciones arquitectónicas y el desarrollo de un prototipo funcional para demostrar la integración efectiva de la IA en estas soluciones.
\newpage

%% Justificacion
\section{Justificación del trabajo de grado}
\label{sec:justificacion}
La creciente necesidad de mejorar y agilizar los procesos en el sector de Seguridad y Salud en el Trabajo ha impulsado la búsqueda de herramientas tecnológicas avanzadas que complementen y potencien las soluciones actuales. Entre estas herramientas, la Inteligencia Artificial (IA) se ha destacado, ofreciendo beneficios significativos en áreas como la automatización y el análisis predictivo.

Organizaciones como la Corporación Talentum reconocen la importancia de integrar tecnologías de IA en sus productos existentes destinados a la seguridad y salud laboral. Esta adaptación no solo responde a un deseo de optimizar procesos y aumentar la eficiencia, sino también a una estrategia para mantenerse competitivos frente a otras empresas que ya están aprovechando la IA en sus soluciones.

A pesar de los potenciales beneficios, la incorporación de IA en este sector presenta varios desafíos. Asegurar la precisión de los datos, garantizar la confiabilidad del análisis y la adaptabilidad de los sistemas existentes son retos prominentes. Adicionalmente, para la Corporación Talentum, se añade el desafío de implementar estas tecnologías tanto en productos establecidos como en proyectos gubernamentales en curso.

La integración de componentes de IA en soluciones software para la Seguridad y Salud en el Trabajo va más allá de una tendencia; se ha convertido en una necesidad palpable. Mediante esta integración, se pretende potenciar el análisis de condiciones laborales, proporcionando recomendaciones más precisas y sugiriendo acciones concretas para mejorar la seguridad de los trabajadores.

En este marco, la propuesta de investigación de maestría adquiere un valor fundamental. El proyecto justifica aún más su relevancia al buscar enfrentar y superar estos obstáculos, con el objetivo de brindar soluciones efectivas y modernas mediante la incorporación de IA en productos y en ejecuciones de proyectos gubernamentales. Así, la investigación no solo beneficia a la Corporación Talentum, sino que también establece principios y marcos que pueden influir en el sector de seguridad y salud laboral en su conjunto.

\newpage

%% Marco conceptual
\section{Marco teórico de referencia y antecedentes}
\subsection{Bases Teóricas}

\subsubsection{Definición y Evolución de la Inteligencia Artificial (IA)}
La Inteligencia Artificial (IA) puede definirse como una rama de la ciencia de la computación que se dedica a desarrollar algoritmos, sistemas y técnicas que permiten a las máquinas aprender y realizar tareas que, hasta hace poco, solo podrían ser realizadas por seres humanos, como el reconocimiento de patrones, la toma de decisiones y la resolución de problemas complejos. En el contexto de la evaluación de la susceptibilidad a movimientos en masa, la IA ha demostrado ser una herramienta invaluable.

En este sentido, Ospina-Gutiérrez y Aristizábal (2021) presentaron una aplicación significativa de la IA en la evaluación de la susceptibilidad a movimientos en masa, específicamente en la cuenca de la quebrada La Miel, en los Andes colombianos. En su estudio, utilizaron diferentes algoritmos de aprendizaje automático para evaluar la capacidad de predicción entre varios modelos, demostrando que los modelos ensamblados tipo boosting superaron significativamente a los modelos paramétricos lineales en términos de desempeño y capacidad de predicción. Este estudio no solo destacó la eficacia de la IA en la predicción y evaluación de áreas susceptibles a movimientos en masa, sino que también enfatizó la importancia de contar con inventarios detallados de movimientos en masa y variables predictoras para el ajuste y desarrollo de modelos útiles para la toma de decisiones y la comprensión del fenómeno \citep{Ospina2021AplicacionMasa}.

\subsubsection{Componentes de Inteligencia Artificial}
Los componentes de la inteligencia artificial, particularmente en el contexto de los Modelos de Lenguaje Pre-entrenados (LLM, por sus siglas en inglés), han demostrado ser herramientas significativas en la era moderna. Una de las incorporaciones prominentes en este ámbito es el desarrollo de chatbots, los cuales hacen uso de sofisticados modelos de lenguaje como el ChatGPT para generar respuestas y mantener conversaciones fluidas con los usuarios \citep{Zamfirescu-Pereira2023WhyPrompts}.

Un aspecto central en la funcionalidad de estos sistemas es el uso de "prompts", que pueden describirse como instrucciones textuales dadas a un LLM para guiar su generación de texto. Los "prompts" pueden ser tanto simples como complejos, integrando elementos variados como preguntas, declaraciones, ejemplos, e instrucciones, que sirven para direccionar las respuestas del modelo de manera más precisa y pertinente hacia una tarea específica \citep{Zamfirescu-Pereira2023WhyPrompts}.

Aunque los "prompts" representan una herramienta vital para potenciar la calidad de las salidas generadas por los LLM, diseñar prompts efectivos puede presentar un desafío considerable, especialmente para individuos que no son expertos en el campo de la inteligencia artificial. Es esencial que los prompts sean confeccionados con un grado específico de detalle para orientar adecuadamente la generación textual, sin restringir excesivamente la capacidad creativa del modelo \citep{Zamfirescu-Pereira2023WhyPrompts}.

Asimismo, los creadores de prompts deben tener en cuenta el contexto específico y la audiencia destinataria, lo que podría requerir una comprensión profunda de la tarea en mano y del modelo de lenguaje subyacente. Este equilibrio delicado significa que diseñar interacciones efectivas con chatbots basados en LLM puede ser una tarea compleja para los no expertos en IA, representando una barrera significativa en la ingeniería de prompts efectivos para los usuarios finales \citep{Zamfirescu-Pereira2023WhyPrompts}.

 A pesar de estos desafíos, la investigación de \citet{Zamfirescu-Pereira2023WhyPrompts} sugiere que los LLM y los prompts tienen un alcance considerable en la sociedad moderna, con implicancias que van más allá de la mera interacción con chatbots. En consecuencia, es imperativo fomentar discusiones y consideraciones más profundas sobre estos componentes y cómo pueden ser integrados de manera más precisa en sistemas que tendrán un impacto palpable en la sociedad.


\subsubsection{Seguridad y salud en el trabajo}
La ``seguridad y salud en el trabajo'' es un campo que abarca medidas y estrategias dirigidas a mantener y promover el bienestar físico y psicológico de los trabajadores \citep{Yaneth2021StrategiesSector,GonzalezDelgado2023AcuteStudy}. En el contexto colombiano, esta área adquiere especial relevancia dada la dinámica de varios sectores industriales, como lo son el de la salud y el de la construcción, que enfrentan desafíos específicos en términos de estrés laboral agudo y la necesidad de implementar estrategias efectivas de capacitación, respectivamente \citep{Yaneth2021StrategiesSector,GonzalezDelgado2023AcuteStudy}.

El estudio realizado por \citep{GonzalezDelgado2023AcuteStudy} arroja luz sobre la prevalencia del estrés agudo entre los trabajadores de la salud en Colombia durante el período 2017-2021. Aunque el contexto es el sector salud, este estudio proporciona una evidencia clara de la necesidad crítica de estrategias y herramientas que puedan ayudar a mitigar estos problemas de estrés, especialmente en sectores en rápida evolución, como lo es el de la tecnología, donde la integración de soluciones de inteligencia artificial podría dar lugar a desafíos similares en términos de bienestar y salud ocupacional \citep{GonzalezDelgado2023AcuteStudy}.

Simultáneamente, el estudio de \citep{Yaneth2021StrategiesSector} destaca la importancia de la formación y capacitación en el sector de la construcción, subrayando la necesidad de desarrollar herramientas y estrategias efectivas para promover la seguridad y la salud en el trabajo. Aunque el estudio está centrado en el sector de la construcción, proporciona insights valiosos que podrían ser aplicables en la Corporación Talentum, a medida que se esfuerza por integrar nuevos productos con inteligencia artificial en sus estructuras existentes, garantizando así que el proceso no solo sea innovador, sino que también promueva un ambiente de trabajo seguro y saludable \citep{Yaneth2021StrategiesSector}.

En resumen, estas investigaciones ofrecen una guía valiosa para la Corporación Talentum en su búsqueda por desarrollar un marco de trabajo que no solo fomente la innovación y la integración de tecnologías avanzadas, sino que también priorice la salud y la seguridad de sus empleados, alineándose así con las mejores prácticas reconocidas en el ámbito colombiano \citep{Yaneth2021StrategiesSector,GonzalezDelgado2023AcuteStudy}.


\subsubsection{Definición de Requerimientos Funcionales}
En el ámbito de la ingeniería del software, es fundamental conceptualizar adecuadamente los términos clave que conforman la estructura de un proyecto. Dentro de este marco, se sitúan los requerimientos, una pieza esencial en el proceso de desarrollo de software, cuya adecuada definición incide directamente en la efectividad y eficacia del producto final \citep{Cipriano2023GPT-3Report,Wu2023AgileDesign}.

En primera instancia, es imperativo clarificar qué son los requerimientos en el contexto de la ingeniería del software. Los requerimientos se categorizan como especificaciones tanto funcionales como no funcionales que los programas de software deben cumplir para satisfacer las necesidades y expectativas de los usuarios y del mercado \citep{Cipriano2023GPT-3Report,Wu2023AgileDesign}. Ahondando en esto, los requerimientos funcionales hacen referencia a las funciones específicas que el software deberá ser capaz de realizar, en tanto que los requerimientos no funcionales se refieren a aspectos más abstractos, como la seguridad, la escalabilidad y la usabilidad del software \citep{Cipriano2023GPT-3Report}.

Centrándonos en los requerimientos funcionales, estos se configuran como instrucciones directas que delinean las funcionalidades que el software debe brindar. En el contexto de una tarea de programación orientada a objetos asignada a GPT-3, por ejemplo, estos requerimientos se traducen en la creación de una jerarquía de clases que simbolizan a los empleados de una empresa de TI, con funciones específicas como la calculación del salario de los empleados y la identificación de la clase jerárquica a la que pertenece cada función \citep{Cipriano2023GPT-3Report}.

En cuanto a la integración de la Inteligencia Artificial (IA) en el desarrollo de productos de software, se destaca el papel fundamental que juega en la optimización de los procesos de diseño. En particular, ChatGPT emerge como una herramienta vital para facilitar una mayor comprensión de las necesidades del usuario y de las dinámicas del mercado \citep{Wu2023AgileDesign}. \citet{Wu2023AgileDesign} profundiza en cómo ChatGPT, como tecnología avanzada de IA, puede asistir inteligentemente a los diseñadores, soportando la toma de decisiones y permitiendo una mejor respuesta a las demandas del mercado, lo que culmina en una aceleración de los ciclos de desarrollo de productos y una mejora en la competitividad del producto.

El proceso para establecer los requisitos del producto, particularmente en el contexto del diseño de Productos Mínimos Viables (MVPs), involucra una metodología que combina la investigación de mercado, la retroalimentación del usuario y el análisis de la competencia. Esta combinación se utiliza para formular una lista exhaustiva de funcionalidades, aspectos de rendimiento y otros factores críticos que deben tenerse en cuenta durante el diseño \citep{Wu2023AgileDesign}. Aquí, ChatGPT se manifiesta como un aliado crucial, facilitando la refinación y optimización de la lista de requerimientos a través de un análisis meticuloso y data-driven, lo que a su vez potencia la eficiencia y calidad del diseño \citep{Wu2023AgileDesign}.

Sin embargo, es igualmente vital considerar que la implementación de IA, como ChatGPT, en el proceso de desarrollo de software, no está exenta de desafíos. Uno de los obstáculos preponderantes radica en la necesidad de entrenar y ajustar continuamente el modelo de IA para que se adapte eficazmente a las necesidades específicas del producto. Además, surgen preocupaciones sobre la interpretabilidad del modelo y la protección de la privacidad de los datos \citep{Wu2023AgileDesign}.

En el caso particular de GPT-3 en la resolución de tareas de programación orientada a objetos, se evidenció que, aunque era capaz de cumplir con los requerimientos funcionales específicos, el código generado no siempre adhería a las mejores prácticas de diseño orientado a objetos, resultando a menudo en código que era desafiante para entender y mantener \citep{Cipriano2023GPT-3Report}. Este escenario señala una limitación significativa, sugiriendo la necesidad de investigaciones adicionales para evaluar cómo estas herramientas de IA pueden ser empleadas de manera más efectiva en entornos educativos y profesionales.

Conclusivamente, la incorporación de IA en el proceso de desarrollo de software promete una revolución significativa, proporcionando una ruta hacia la eficiencia mejorada, una calidad de diseño superior, y una innovación revolucionaria, especialmente en el ámbito del desarrollo de Productos Mínimos Viables (MVPs). Esta transformación es evidente en la utilidad del modelo de procesamiento de lenguaje natural ChatGPT, el cual facilita la comprensión profunda de las necesidades de los usuarios y las tendencias del mercado, optimizando, así, la eficiencia y la calidad del diseño de productos \citep{Wu2023AgileDesign}.

Al considerar los requerimientos funcionales, estos hacen referencia a las especificaciones que delinean las funciones que un software debe realizar. Estos son cruciales en la fase de definición de cualquier proyecto de desarrollo de software, ya que establecen las operaciones fundamentales que deben ser implementadas para satisfacer las necesidades de los usuarios finales. En el contexto de los proyectos abordados por ChatGPT, los requerimientos funcionales pueden involucrar la identificación de los usuarios objetivo y la determinación de las funcionalidades centrales que servirán para mejorar la experiencia del usuario y satisfacer las demandas del mercado \citep{Wu2023AgileDesign}.

Es imperativo que los diseñadores y desarrolladores estén conscientes de las implicancias de la integración de IA, tanto en términos de oportunidades como de desafíos. En particular, se deben tener en cuenta consideraciones clave como la interpretabilidad del modelo y la protección de la privacidad de los datos \citep{Wu2023AgileDesign}.

Por otro lado, en el contexto educativo, la herramienta de generación de lenguaje natural desarrollada por OpenAI, GPT-3, ha mostrado potencial para ayudar a resolver tareas de programación orientada a objetos. Si bien GPT-3 puede interpretar y gestionar requerimientos funcionales directos, tiene la tendencia de no proporcionar la mejor solución en términos de diseño orientado a objetos, a menudo resultando en código que puede ser difícil de entender y mantener \citep{Cipriano2023GPT-3Report}. Esto señala la necesidad de más investigaciones y adaptaciones para mejorar su utilidad en este contexto, especialmente en lo que respecta a adherirse a las mejores prácticas de diseño orientado a objetos y facilitar la creación de código que sea tanto funcional como sostenible \citep{Cipriano2023GPT-3Report}.

En vista de los hallazgos actuales, es evidente que nos encontramos en el umbral de una era de innovación y eficiencia mejorada en el diseño y desarrollo de software, con la IA desempeñando un papel crucial en este avance. No obstante, resulta fundamental profundizar en la exploración de la inteligencia artificial en campos tales como la seguridad y salud en el trabajo. Es esencial evaluar cómo estas nuevas tecnologías pueden contribuir significativamente no solo en otros sectores, sino también en la mejora y garantía de la seguridad y salud ocupacional, fomentando así la integración de innovaciones tanto en productos existentes como en nuevos desarrollos en este ámbito.

\subsection{Estado del Arte}
El estado del arte que se presenta a continuación pretende evidenciar la confluencia de dos campos cruciales: la salud y bienestar de los trabajadores en su entorno laboral y la incorporación de la inteligencia artificial en este ámbito. La seguridad y salud en el trabajo no solo se centra en la prevención de accidentes, sino también en la salud integral del trabajador, lo que incluye aspectos médicos y de atención sanitaria. Es por esta razón que la literatura a explorar abordará temáticas que entrelazan la atención médica con aplicaciones de IA. Estas integraciones emergen como piezas fundamentales para el desarrollo y mejoramiento de productos y sistemas destinados a salvaguardar y promover la salud de los trabajadores en sus respectivos espacios laborales. Así, el lector encontrará en este estado del arte información sobre productos de IA, así como de componentes arquitectónicos e integraciones actuales aplicadas a la salud o salud ocupacional.

\subsubsection{ChatGPT en el Ámbito Médico: Implicaciones, Potencialidades y Retos en la Atención de Salud Laboral}
El mundo ha sido testigo de cómo ChatGPT, creado por OpenAI, ha causado un profundo impacto en diversos campos, incluido el de la atención médica. En solo dos meses después de su lanzamiento, ChatGPT atrajo a 100 millones de usuarios, superando incluso a plataformas previamente populares como TikTok \citep{Kleesiek2023AnOnly}. Esto demuestra su gran influencia y su potencial en diversos sectores, incluido el de la salud laboral.

La esencia de ChatGPT reside en su tecnología subyacente: un modelo de lenguaje grande (LLM) conocido como generative pretrained transformer (GPT-3.5) entrenado con 175 mil millones de parámetros \citep{Kleesiek2023AnOnly}. Estos LLMs, originados en el procesamiento de lenguaje natural, han demostrado ser modelos fundamentales que pueden adaptarse a una amplia variedad de tareas debido a sus capacidades de aprendizaje con pocos ejemplos y transferencia de conocimiento \citep{Kleesiek2023AnOnly}.

Sin embargo, mientras que ChatGPT ha sorprendido al mundo con su habilidad conversacional y AI, ha surgido una distinción crítica entre la capacidad de conversación general de ChatGPT y las aplicaciones médicas específicas. La programación y el entrenamiento de ChatGPT están diseñados para conversaciones generales y no para soporte diagnóstico o recomendaciones de tratamiento \citep{Kleesiek2023AnOnly}. Esta delimitación es esencial, especialmente en el ámbito de la salud laboral, donde la precisión y la fiabilidad de la información son vitales.

La discusión sobre la función de ChatGPT en la atención médica, particularmente en el contexto laboral, se centra en dos aspectos críticos: el uso previsto versus el uso real y las expectativas de los desarrolladores en contraposición a las de los usuarios finales \citep{Kleesiek2023AnOnly}. A pesar de que ChatGPT siempre aclara que no es un profesional de la salud, las posibles implicaciones de su uso en la atención médica laboral plantean preguntas sobre la responsabilidad y la precisión del contenido que genera.

Las tecnologías disruptivas como ChatGPT ofrecen tanto amenazas como oportunidades. En el mejor de los casos, pueden surgir sinergias entre humanos y computadoras, como la combinación de capacidades humanas y computacionales para lograr objetivos más amplios. Sin embargo, el peligro radica en confiar ciegamente en la tecnología, lo que puede llevar a la desinformación, especialmente en un área tan crítica como la atención médica laboral \citep{Kleesiek2023AnOnly}.

Es innegable que LLMs como ChatGPT tienen un vasto potencial en la atención médica. Desde la generación de texto para completar informes clínicos hasta la interpretación y explicación de otros algoritmos de AI, las aplicaciones son vastas \citep{Kleesiek2023AnOnly}. Sin embargo, es esencial que estos desarrollos se realicen con precaución, especialmente en el ámbito laboral, donde las decisiones basadas en la información proporcionada pueden tener consecuencias significativas para la salud y el bienestar de los trabajadores.

En conclusión, mientras ChatGPT y tecnologías similares ofrecen posibilidades emocionantes en el ámbito de la salud laboral, es crucial que se utilicen con discernimiento, y se comprendan plenamente sus limitaciones y potencialidades. El futuro puede ser prometedor, pero es esencial que, como sociedad, guiemos su desarrollo de manera responsable \citep{Kleesiek2023AnOnly}.



\subsubsection{La Integración de Inteligencia Artificial en la Salud Ocupacional: Una Evaluación Crítica de ChatGPT en Escenarios Clínicos y de Investigación}
El avance tecnológico ha llevado a la incorporación de la inteligencia artificial (IA) en múltiples dominios de la atención médica. Un análisis reciente realizado \citet{Cascella2023EvaluatingScenarios} exploró la viabilidad de una de estas herramientas de IA, específicamente ChatGPT, en diferentes escenarios clínicos y de investigación. Esta exploración es especialmente relevante en el contexto de la salud y bienestar de los trabajadores, ya que la seguridad y salud laboral no solo comprenden la prevención de accidentes, sino también la salud integral del trabajador, lo que abarca aspectos médicos y de atención sanitaria.

\citet{Cascella2023EvaluatingScenarios} examinaron la capacidad de ChatGPT para comprender y razonar sobre temas de salud pública, especialmente en relación con el concepto de seniority y cómo se mide objetivamente la edad biológica de una persona. El estudio reveló que el chatbot podía proporcionar definiciones precisas, categorizar a los adultos mayores en subgrupos según su edad, y mencionar métodos comunes para estudiar la senioridad desde una perspectiva biológica, como el desarrollo dental y esquelético, la longitud de los telómeros y la función cognitiva, entre otros \citep{Cascella2023EvaluatingScenarios}.

Más aún, ChatGPT demostró su capacidad para citar estudios clínicos relevantes que respaldaban sus respuestas, lo que sugiere su potencial en la exploración de la literatura y la generación de nuevas hipótesis de investigación. Sin embargo, \citet{Cascella2023EvaluatingScenarios} también identificaron posibles mal uso de la herramienta, como la generación de noticias falsas o información errónea, y el potencial de ChatGPT para reproducir sesgos presentes en los datos con los que fue entrenado.

Dada la importancia de la salud ocupacional y la necesidad de sistemas y productos que mejoren la salud de los trabajadores, las aplicaciones de IA como ChatGPT emergen como herramientas cruciales. Aun así, es esencial que la comunidad científica comprenda sus límites y capacidades para garantizar su uso efectivo y seguro en contextos clínicos y de investigación \citep{Cascella2023EvaluatingScenarios}.

En conclusión, la incorporación de herramientas de IA en la salud y bienestar de los trabajadores presenta un horizonte prometedor, pero también plantea desafíos significativos. Los hallazgos de \citet{Cascella2023EvaluatingScenarios} proporcionan una visión crítica de una de estas herramientas, ChatGPT, resaltando su potencial y limitaciones en escenarios de atención médica.






% Esta sección da cuenta del estado en el que se encuentra la investigación sobre el tema que se está explorando con el proyecto de grado. Tiene como objetivo revisar y analizar el conocimiento acumulado alrededor del problema, y evidenciar cuál es el estado actual de la solución a un problema respecto al problema que se desea abordar. 

% Esta sección presenta trabajos previos (estudios o implementaciones) que abordan el problema de forma similar, da confianza sobre el conocimiento del autor de referentes anteriores así como permite que no se repitan estudios sobre asuntos explorados previamente.

% \textbf{Nota:} \textit{En el anteproyecto este análisis puede ser más superficial pero a medida que lo haga mejor podrá reutilizar más para su documento final.}

% \subsubsection*{¿Qué incluir?}
% Piense en los siguientes temas:
% \begin{itemize}
%     \item ¿Cómo se ha resuelto el problema?
%     \item ¿Quienes han resuelto el problema?
%     \item ¿Qué aspectos técnicos económicos, culturales normativas estándares se han tenido en cuenta?
%     \item ¿El problema ha sido resuelto en otro contexto?
% \end{itemize}

% \subsubsection*{¿Qué NO debe incluir?}
% \begin{itemize}
%     \item NO incluya ideas propias o reflexiones respecto a cómo solucionar el problema. Facts only. 
%     \item NO describa información de otros trabajos o problemáticas que usted NO va a abordar 
% \end{itemize}

% \subsubsection*{¿Cómo organizarlo?}
% \begin{itemize}
%     \item Definición de un conjunto de criterios que van a usarse para comparar los trabajos de otros
%     \item Una descripción corta de cada propuesta - previas de solución del problema. Resalte en cada una ventajas y desventajas. 
%     \item  Indique de forma clara: ¿Por qué las propuestas y soluciones revisadas no sirven en el contexto del estudio y porque no resuelven la pregunta planteada en su proyecto de investigación?
%     \item \textbf{DESEABLE}: haga tablas o gráficas que presenten cuál es el vacío que tiene la situación actual. 
% \end{itemize}



\newpage

%% Metodología
\section{Metodología de la investigación}
Este capítulo describe la metodología de investigación adoptada para guiar la integración de la Inteligencia Artificial en soluciones de software para el sector de la Seguridad y Salud en el Trabajo. Se articula en varias fases, cada una con actividades específicas orientadas a cumplir los objetivos del proyecto.

\subsection{Fase 1: Recopilación de buenas prácticas}
La primera fase del proyecto se concentra en la identificación de las mejores prácticas para la incorporación de la Inteligencia Artificial en el software, con un enfoque particular en la gestión de los recursos computacionales, la selección de herramientas y librerías, y la elección de modelos de IA adecuados para el sector de la Seguridad y Salud en el Trabajo.

\subsubsection{Actividad 1.1: Revisión teórica de literatura existente}
Se llevará a cabo una investigación de la literatura existente para recopilar estrategias en la integración de la IA. Esta actividad implica un análisis de documentos académicos, reportes técnicos y estudios de caso que ilustren buenas prácticas en la gestión de recursos computacionales, la optimización de herramientas y librerías, y la implementación de modelos de IA.

\subsubsection{Actividad 1.2: Consolidación de prácticas para la Integración de IA}
Posterior a la revisión teórica, se consolidará un compendio de prácticas, sirviendo de referencia para la arquitectura del software y la selección de componentes de Inteligencia Artificial. Esta compilación ayudara como guía o referencia en las decisiones de las etapas de diseño y desarrollo inicial, permitiendo un lineamiento para la integración de la IA en el ámbito de SST.





\subsection{Fase 2: Análisis de Componentes Arquitectónicos}
Esta fase se enfoca en el estudio de los componentes arquitectónicos que permitirán una integración de la Inteligencia Artificial en el ámbito del software. La atención se centra en tres estilos arquitectónicos fundamentales: monolítico, microservicios y basado en servicios.

\subsubsection{Actividad 2.1: Revisión de elementos arquitectónicos}
Se procederá a identificar y analizar elementos arquitectónicos fundamentales para la integración de la Inteligencia Artificial. Se explorarán diversas fuentes para resaltar las características y beneficios de estos elementos, estableciendo un fundamento para la futura incorporación de componentes de IA en sistemas de software, con especial énfasis en su aplicabilidad en SST.

\subsubsection{Actividad 2.2: Categorización de elementos arquitectónicos}
Esta actividad se centra en la categorización de los componentes arquitectónicos seleccionados. El objetivo es desarrollar una caracterización que facilite la integración armoniosa de dichos elementos en la fase de diseño de software, promoviendo la incorporación sinérgica de soluciones de Inteligencia Artificial en plataformas especializadas en SST.





\subsection{Fase 3: Definición de Contexto en SST}
Esta fase se dedica a especificar teóricamente el contexto de la Seguridad y Salud en el Trabajo (SST) para ajustar de manera precisa la Inteligencia Artificial a los requerimientos priorizados del sector en la Corporación Talentum.

\subsubsection{Actividad 3.1: Análisis Teórico de Contexto para la IA en SST}
Se llevará a cabo un estudio teórico para entender y determinar los contextos dentro de los cuales la Inteligencia Artificial debería operar para mejorar la precisión y relevancia de sus respuestas en el ámbito de la SST. Se analizarán los factores y variables que influyen en la interacción con sistemas de IA, con el objetivo de que las respuestas generadas correspondan a las particularidades y requerimientos del sector.




\subsection{Fase 4: Adaptaciones Arquitectónicas}
Esta fase se centra en la revisión de la literatura sobre estilos arquitectónicos como monolíticos, microservicios y basados en servicios, con el fin de recopilar información pertinente que guíe las adaptaciones necesarias en las arquitecturas de software existentes para la integración de la IA.

\subsubsection{Actividad 4.1: Evaluación de cambios arquitectónicos}
Se llevará a cabo una evaluación de las adaptaciones arquitectónicas que resulten más pertinentes para sistemas preexistentes, tomando como referencia los componentes arquitectónicos identificados en la fase anterior. Esta revisión tiene como objetivo determinar las modificaciones necesarias que permitan la incorporación de la Inteligencia Artificial.

\subsubsection{Actividad 4.2: Revisión teórica sobre estilos de arquitectura de integración.}
Se profundizará en la revisión de literatura, publicaciones y documentos técnicos sobre arquitectura de integración. Se estudiarán los estilos y tácticas más actuales y su aplicabilidad al contexto del proyecto, dentro del marco del proyecto se revisarán los estilos monolíticos, microservicios y basado en servicios.





\subsection{Fase 5: Desarrollo y validación del prototipo}
Esta fase se concentra en la implementación y posterior verificación de un prototipo que ejemplifique la aplicación práctica del marco de trabajo desarrollado. Este prototipo, será construido con la colaboración de la Corporación Talentum para el levantamiento de requerimientos, funcionará como un caso de prueba para comprobar el marco propuesto en atender las necesidades específicas del sector de SST.

\subsubsection{Actividad 5.1: Análisis y recolección de requerimientos}
Se efectuará una comprensión de las necesidades y expectativas de la Corporación Talentum. Esta actividad se enfocará en recoger y analizar los requisitos específicos para el prototipo de software, garantizando que la solución final esté perfectamente alineada con las demandas del sector de SST.

\subsubsection{Actividad 5.2: Priorización de requerimientos}
A través de talleres colaborativos, se priorizarán los requerimientos del prototipo. Esta priorización es para desarrollar una hoja de ruta estratégica para el proyecto y para prever retos potenciales en la fase de implementación.

\subsubsection{Actividad 5.3: Diseño y desarrollo de un prototipo aplicando el marco de trabajo}
Con los requerimientos ya establecidos, se dará inicio al proceso de diseño y desarrollo del prototipo. Esta fase aplicará el marco de trabajo diseñado para construir un caso de estudio que demuestre la aplicabilidad del marco en el cumplimiento de los objetivos del proyecto, particularmente en la integración de IA en soluciones de SST.

\subsubsection{Actividad 5.4: Evaluación del uso del marco de trabajo en el prototipo}
Finalmente, se evaluará cómo la aplicación del marco de trabajo ha influenciado positivamente el desarrollo del prototipo, especialmente en lo que respecta a la incorporación de la IA. Se verificará que el marco de trabajo no solo ha facilitado el proceso de desarrollo, sino que también ha guiado la atención a los requerimientos de IA, garantizando resultados alineados con las expectativas del proyecto.








% Para garantizar una incorporación efectiva de componentes de Inteligencia Artificial en soluciones software dirigidas al sector de seguridad y salud en el trabajo, se establece una serie de fases y actividades sistemáticas. Estas fases están diseñadas para atender los requerimientos específicos y cumplir con los objetivos propuestos para el proyecto.

% \subsection{Fase 1: Análisis e identificación de Requerimientos}
% Antes de iniciar, es esencial comprender a fondo las necesidades y expectativas. En el ámbito de las soluciones software, es crucial identificar los requerimientos en este caso de la Corporación Talentum. En esta fase, se dedica un esfuerzo para identificar las especificaciones relevantes. El enfoque aquí no solo es recolectar la información, sino también entender el contexto detrás de cada requerimiento para que el prototipo final pueda satisfacer de manera óptima las demandas del sector.

% \subsubsection{Actividad 1.1: Obtención de requerimientos y características mínimas del prototipo.}
% Se organizará y conducirá una serie de reuniones estructuradas con profesionales de la Corporación Talentum que tienen experiencia en proyectos relacionados con la seguridad y salud en el trabajo. A través de estas reuniones, se buscará recopilar e identificar las especificaciones detalladas, características y funciones que el prototipo debe incluir. El objetivo es definir un conjunto de requerimientos que respondan a las necesidades actuales del sector.

% \subsubsection{Actividad 1.2: Evaluación de la prioridad del primer requerimiento del prototipo en Corporación Talentum.}
% Mediante sesiones de trabajo interactivas, se evaluarán las necesidades y prioridades que la Corporación Talentum considera más urgentes. Esta evaluación permitirá determinar cuál requerimiento deberá abordarse primero y establecer una hoja de ruta para el desarrollo del prototipo. Además, se identificarán posibles desafíos o limitaciones asociados con la implementación de este requerimiento.

% \subsubsection{Actividad 1.3: Documentación de la arquitectura de software y componentes necesarios.}
% Con base en los requerimientos identificados, se procederá a diseñar una arquitectura de software coherente que permita una integración eficiente de las diferentes funcionalidades. Esta arquitectura se documentará en detalle, incluyendo los componentes y módulos necesarios, sus interrelaciones, y se asegurará que sea modular, interoperable y escalable para adaptarse a futuras necesidades.

% \subsection{Fase 2: Diseño y Desarrollo del Prototipo}
% Una vez establecidos los requerimientos, el siguiente paso es el desarrollo del proyecto. En esta fase, se traza la ruta para convertir las especificaciones teóricas en un prototipo. Es un proceso iterativo, donde el diseño y el desarrollo se llevan a cabo simultáneamente, permitiendo adaptaciones basadas en descubrimientos realizados durante el proceso. La colaboración y la comunicación constante con los stakeholders, en este caso la Corporación Talentum, es esencial para garantizar que el producto final esté alineado con las visiones y necesidades iniciales.

% \subsubsection{Actividad 2.1: Selección de tecnología para el desarrollo del prototipo.}
% Se llevará a cabo un análisis exhaustivo sobre las tecnologías y herramientas actuales en el mercado que sean relevantes para el proyecto. Esta investigación contemplará aspectos como compatibilidad, escalabilidad, soporte técnico y costos asociados. Tras este análisis, se decidirá sobre las tecnologías que ofrecen la mayor adaptabilidad y rendimiento para el prototipo.

% \subsubsection{Actividad 2.2: Inicio del desarrollo del prototipo basado en los requerimientos de Corporación Talentum.}
% Se iniciará el desarrollo del prototipo, respetando las especificaciones y requerimientos recabados. Durante esta fase, se mantendrá una comunicación constante con la Corporación Talentum, asegurando que el desarrollo esté alineado con sus expectativas y necesidades. Se realizarán iteraciones y ajustes según el feedback recibido.

% \subsection{Fase 3: Definición del Marco de Trabajo y Componentes Tecnológicos}
% Con el diseño y desarrollo en marcha, es esencial establecer un marco de trabajo que guíe las integraciones tecnológicas y facilite la escalabilidad del proyecto. Esta fase se concentra en definir y documentar el marco de trabajo con la arquitectura de software y las herramientas tecnológicas que se utilizarán. La investigación teórica en esta etapa ayuda a garantizar que las decisiones tomadas estén fundamentadas en prácticas probadas y estén alineadas con las tendencias actuales de la industria.

% \subsubsection{Actividad 3.1: Revisión teórica sobre estilos de arquitectura de integración.}
% Se profundizará en la revisión de literatura, publicaciones y documentos técnicos sobre arquitectura de integración. Se estudiarán los estilos y tácticas más actuales y su aplicabilidad al contexto del proyecto.

% \subsubsection{Actividad 3.2: Revisión teórica sobre patrones de integración.}
% Se consultará literatura especializada y se analizarán estudios de caso para obtener un panorama claro sobre los patrones de integración que han demostrado ser efectivos en proyectos similares. Se buscará identificar y documentar patrones que puedan ser replicados o adaptados para el proyecto en cuestión.

% \subsubsection{Actividad 3.3: Revisión teórica de especificación de contextos para interoperar con la IA en el ámbito de seguridad y salud en el trabajo.}
% Se realizará una revisión teórica exhaustiva para comprender y definir los contextos necesarios que mejoren la precisión y relevancia de las respuestas proporcionadas por la Inteligencia Artificial (IA) en el ámbito de la seguridad y salud en el trabajo. Se enfocará en identificar y especificar los parámetros y variables que deben ser considerados al interactuar con sistemas de IA, asegurando que las respuestas generadas estén alineadas con las necesidades y expectativas de la Corporación Talentum. Además, se explorarán las mejores prácticas y estrategias para estructurar y formular prompts de manera efectiva, contribuyendo así a una interacción más eficiente y precisa con las herramientas de IA.

% \subsection{Fase 4: Validación y Pruebas}
% Esta fase se dedica a validar que el prototipo no solo cumpla con las especificaciones técnicas, sino que también satisfaga a los usuarios. A través de una serie de pruebas, se identificarán y rectificarán las áreas problemáticas, garantizando que el prototipo final cumpla con las definiciones previas.

% \subsubsection{Actividad 4.1: Pruebas unitarias.}
% Se realizarán pruebas unitarias sobre los componentes de software. Esto garantizará que cada función o módulo opere correctamente y cumpla con las especificaciones técnicas previamente definidas. Además, se buscará identificar y corregir cualquier fallo o incoherencia en esta etapa temprana del proceso de prueba.

% \subsubsection{Actividad 4.2: Pruebas de Aceptación.}
% Se organizarán sesiones en las que representantes de la Corporación Talentum interactuarán con el prototipo. Estas pruebas de aceptación permitirán evaluar funcionalidad. A través de estas interacciones, se socializará al equipo de la Corporación Talentum todo el prototipo.



\newpage

%% Recursos
\section{Recursos a emplear}
\label{sec:recursos}
\subsection{Recursos Humanos}
\begin{enumerate}
    \item Director: Ceballos Argote, Oscar Orlando. Ingeniero de Sistemas, Magister en Ingenier\'ia de Sistemas y Computaci\'on, Doctor en Ciencias de la Computaci\'on. Actualmente se desempe\~na como Ingeniero de Datos en Globant y est\'a asociado a la cuenta de Deloitte Canad\'a. Se ha desempe\~nado como profesor hora c\'atedra y tiempo completo en programas de pregrado y postgrado en la Universidad de Nari\~no y en la Universidad del Valle. Sus conocimientos e investigaci\'on se enfocan prinicipalmente en el procesamiento de Big Data.
    \item Estudiante: Fabian Andres Caicedo Cuellar. Ingeniero de Sistemas, Estudiante de Maestría en Ingeniería de Software, Universidad Javeriana Cali.
\end{enumerate}

\subsection{Recursos Bibliográficos y de investigación}
\begin{enumerate}
    \item Se harán uso de recursos bibliográficos, tanto físicos como digitales, de la Biblioteca de la Pontificia Universidad Javeriana Cali.

    \item Se utilizarán diversas plataformas en línea, tales como Google Scholar y foros especializados en desarrollo de software. Además, se consultará la documentación de software de código abierto para obtener información relevante sobre los temas tratados en el proyecto.
\end{enumerate}

\subsection{Recursos tecnológicos}
\begin{enumerate}
    \item Se utilizará los servicios de AWS de la capa gratuita para aprovisionar la infraestructura cloud.
    \item Se emplearán equipos de cómputo Mac Book Pro con procesadores M2, basados en la arquitectura ARM.
\end{enumerate}
\newpage

%% Cronograma
\section{Cronograma de actividades}
% \begin{ganttchart}[vgrid, hgrid, chart element start border=right]{1}{12}
% \gantttitle{Mes 1}{4} \\
% \gantttitlecalendar{week} \\[grid]
% \ganttgroup{Fase 1}{0}{4} \\
% \ganttbar{Actividad 1.1}{0}{2} \\
% \ganttbar{Actividad 1.2}{2}{4} \\
% % \ganttmilestone{Milestone 1}{11}
% \end{ganttchart}

% \begin{ganttchart}[
% hgrid,
% vgrid,
% x unit=4mm,
% time slot format=isodate
% ]{2012-12-25}{2013-02-01}
% \gantttitlecalendar{year, month, day, week=3, weekday} \\
% \ganttbar{}{2013-01-14}{2013-01-17}
% \end{ganttchart}

\begin{figure}[h!]
    \centering
    \begin{ganttchart}[
        hgrid,
        % vgrid={*{6}{draw=none}, dotted},
        vgrid={*{1}{dotted}},
        x unit=0.55cm,
        y unit chart=0.8cm,
        title/.append style={fill=gray!30},
        title label font=\tiny,    
        % title label node/.append style={anchor=center},
        group/.append style={draw=black, fill=green!50},
        bar/.append style={fill=blue!30, draw=blue!50},
        bar height=0.4
    ]{1}{24}
        % Meses
        \gantttitle{Mes 1}{4} 
        \gantttitle{Mes 2}{4} 
        \gantttitle{Mes 3}{4} 
        \gantttitle{Mes 4}{4} 
        \gantttitle{Mes 5}{4} 
        \gantttitle{Mes 6}{4} \\
        % Semanas
        \gantttitle{Semanas}{24} \\
        \gantttitlelist{1,2,3,4,5,6,7,8,9,10,11,12,13,14,15,16,17,18,19,20,21,22,23,24}{1} \\
        % Fase 1
        \ganttgroup{Fase 1}{1}{4} \\
        \ganttbar{Actividad 1.1}{1}{2} \\
        \ganttbar{Actividad 1.2}{3}{4} \\
        \ganttbar{Actividad 1.3}{1}{4}\ganttnewline[thick, blue]
        
        % Fase 2
        \ganttgroup{Fase 2}{5}{12} \\
        \ganttbar{Actividad 2.1}{5}{6} \\
        \ganttbar{Actividad 2.2}{6}{12}\ganttnewline[thick, blue]
        
        % Fase 3
        \ganttgroup{Fase 3}{13}{20} \\
        \ganttbar{Actividad 3.1}{13}{16} \\
        \ganttbar{Actividad 3.2}{15}{18} \\
        \ganttbar{Actividad 3.3}{17}{20}\ganttnewline[thick, blue]
        
        % Fase 4
        \ganttgroup{Fase 4}{7}{20} \\
        \ganttbar{Actividad 4.1}{7}{12} \\
        \ganttbar{Actividad 4.2}{13}{20}\ganttnewline[thick, blue]
        
        % Documentación
        % \ganttgroup{Documentación}{1}{24} \\
        \ganttbar{Documentación}{1}{24}
                
    \end{ganttchart}
    
    \caption{Cronograma}
    \label{fig:gantt_chart}
\end{figure}



\newpage

\section{Referencias Bibliográficas}
% Se deben presentar de forma rigurosa y completa las referencias bibliográficas utilizadas en el documento (No incluir bibliografía no referenciada en el documento). Se debe seguir una sola norma de referenciación en todo el documento, ej. IEEE, Harvard o APA.

% Utilizar en lo posible bibliografía reciente de fuentes confiables y en inglés(libros, artículos científicos, etc.). Evitar utilizar fuentes no confiables como blogs, Wikipedia, o documentos sin autor.

%\bibliographystyle{IEEEtran}
\bibliographystyle{apalike}

\label{sec:bibliografia}
\bibliography{references.bib}

\newpage

%% Cronograma
\section{Glosario de Términos}
\begin{description}
    \item [Inteligencia Artificial (IA):] Se refiere a la simulación de procesos de inteligencia humana por máquinas, especialmente sistemas informáticos. Estos procesos incluyen el aprendizaje (adquisición de información y reglas para usar la información), el razonamiento (usar reglas para llegar a conclusiones aproximadas o definitivas) y la auto-corrección.
    
    \item [Marco de Trabajo (Framework):] Es una plataforma estandarizada y, a menudo, un conjunto de herramientas y bibliotecas que facilitan el desarrollo y la gestión de aplicaciones y sistemas de software. Los marcos de trabajo proporcionan una base sobre la cual se pueden desarrollar aplicaciones, asegurando consistencia, eficiencia y a menudo incorporando prácticas recomendadas.

    \item [Prototipo Funcional:] En el desarrollo de software, se refiere a una versión inicial o modelo de un programa que tiene la funcionalidad esencial para demostrar un concepto o proceso. Se utiliza para probar y refinar características antes de desarrollar una versión completa o final del software.
    
    \item [Arquitectura de Software:] Se refiere a la estructura y diseño de un sistema de software, incluyendo los componentes del sistema, las relaciones entre esos componentes y las interfaces mediante las cuales interactúan. La arquitectura de software sirve como un plan o esquema que describe cómo se integran y funcionan juntas las diferentes partes de un sistema.

    \item [Requerimientos Funcionales:] Son declaraciones detalladas de las capacidades que debe tener un sistema, las interacciones que debe soportar y las actividades que debe poder realizar. Específicamente, describen lo que hace el sistema, como procesar datos, interactuar con el usuario u operar con otros sistemas.

    \item [Seguridad y Salud en el Trabajo:] Se refiere al conjunto de disciplinas y medidas que buscan proteger y mejorar el bienestar físico, mental y social de los trabajadores en sus lugares de trabajo. Esta área se enfoca en anticipar, reconocer, evaluar y controlar aquellos factores en el ambiente laboral que pueden causar enfermedad o lesiones, y afectar el bienestar de los trabajadores y sus comunidades.

    \item [Prompt:] En el contexto de programación y sistemas informáticos, un prompt es una secuencia de caracteres que se utiliza en una interfaz de usuario para indicar la disposición para recibir entradas del usuario. También puede referirse a la invitación visual en interfaces de línea de comandos que indica al usuario que el sistema está listo para recibir comandos. En el contexto de modelos de lenguaje como OpenAI, un prompt es la entrada dada al modelo para generar una respuesta o continuación.
\end{description}


\end{description}

\end{document}